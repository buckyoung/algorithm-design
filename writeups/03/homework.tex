 
%
% Homework Details
%   - Title
%   - Due date
%   - Class
%   - Section/Time
%   - Instructor
%   - Author
%

\newcommand{\hmwkTitle}{Greedy Algorithms}
\newcommand{\hmwkProblems}{6 and 7}
\newcommand{\hmwkDueDate}{September 3, 2014}
\newcommand{\ClassName}{Algorithm Design}
\newcommand{\ClassNumber}{CS 1510}
\newcommand{\hmwkAuthorName}{Buck Young and Rob Brown}



%
% Basic Document Settings
%

\documentclass{article}

\usepackage{fancyhdr}
\usepackage{extramarks}
\usepackage{amsmath}
\usepackage{amsthm}
\usepackage{amsfonts}
\usepackage{tikz}
\usepackage[plain]{algorithm}
\usepackage{algpseudocode}
\usepackage{amssymb}

\usetikzlibrary{automata,positioning}


\topmargin=-0.45in
\evensidemargin=0in
\oddsidemargin=0in
\textwidth=6.5in
\textheight=9.0in
\headsep=0.25in\newcommand{\hmwkClassTime}{Section A}
\linespread{1.1}

\renewcommand\headrulewidth{0.4pt}
\renewcommand\footrulewidth{0.4pt}
\setlength\parindent{0pt}

%
% Create Problem Sections
%

\newcommand{\enterProblemHeader}[1]{
    \nobreak\extramarks{}{Problem \arabic{#1} continued on next page\ldots}\nobreak{}
    \nobreak\extramarks{Problem \arabic{#1} (continued)}{Problem \arabic{#1} continued on next page\ldots}\nobreak{}
}

\newcommand{\exitProblemHeader}[1]{
    \nobreak\extramarks{Problem \arabic{#1} (continued)}{Problem \arabic{#1} continued on next page\ldots}\nobreak{}
    \stepcounter{#1}
    \nobreak\extramarks{Problem \arabic{#1}}{}\nobreak{}
}

\setcounter{secnumdepth}{0}
\newcounter{partCounter}
\newcounter{homeworkProblemCounter}
\setcounter{homeworkProblemCounter}{1}
\nobreak\extramarks{Problem \arabic{homeworkProblemCounter}}{}\nobreak{}

\newenvironment{homeworkProblem}{
    \section{ }
    %\setcounter{partCounter}{1}
    %\enterProblemHeader{homeworkProblemCounter}
}{
    \exitProblemHeader{homeworkProblemCounter}
}



% 
% Header and Footer definition
%

\pagestyle{fancy}
\lhead{\ClassNumber\ - \ClassName}
\chead{\hmwkTitle}
\rhead{Problems \hmwkProblems}
\lfoot{\lastxmark}
\cfoot{\thepage}


%
% Title Page
%

\title{
    \vspace{2in}
	\textmd{\textbf{\ClassNumber}} \\
    \textmd{\textbf{\ClassName}} \\    
    \normalsize\vspace{0.1in}\small{\hmwkTitle} \\
    \normalsize\vspace{0.1in}\small{Problems \hmwkProblems} \\
	\normalsize\vspace{0.1in}\small{Due \hmwkDueDate}    \\
    \vspace{3in}
}

\author{\textbf{\hmwkAuthorName}}
\date{}

\renewcommand{\part}[1]{\textbf{\large Part \Alph{partCounter}}\stepcounter{partCounter}\\}






% 	% 	%	%	%	%	%
%	Document Start 		%
% 	% 	% 	% 	% 	% 	%

\begin{document}

\maketitle

\pagebreak




\begin{homeworkProblem}
\centerline{\textbf{Problem 6}}
\leavevmode
\\ \\
\textbf{(a)}
\\
\textbf{Input:} An ordered sequence of words $S = \{w_1, w_2, ... , w_n\}$ where the length of the $i^{th}$ word is $w_i$.
\\ \\
\textbf{Output:} Total penalty P as described by the problem statement ($\Sigma P_i$ for each $P_i = K_i - L$) such that P is minimized. 
\\ \\
\textbf{Theorem:} The given greedy algoirthm G correctly solves this problem.
\\ \\
\textbf{Proof:} 
Assume to reach a contradiction that there exists some input I on which G produces an unacceptable output. Let OPT be the optimal solution that agrees with G for the most number of steps. Let the first disagreance be word $w_{L_k}$ on line $k$. The only way OPT can disagree with G is if OPT moves $w_{L_k}$ to a new line and G does not (obviously G wont fail to place a word on a line that it is able to).
\\ \\ \\ \\ \\ \\ \\ \\ \\ \\ \\
Let OPT' be a modified version of OPT, such that with the first word on each new line ($w_{L_k}, w_{L_{k+1}}, ..., w_{L_n}$) each moved to the previous line if possible. The logic governing moving the first word on the $i^{th}$ line up to the $(i-1)^{th}$ line is as follows.
\\ \\
\begin{tabbing}
\hspace*{2cm}\=\hspace*{3cm}\= \kill
\center{if $w_i <= (L - K_{i-1})$:

}
\end{tabbing}




\end{homeworkProblem}




\begin{homeworkProblem}
\centerline{\textbf{Problem 5}}
\leavevmode
\\ \\
\textbf{(a)}
\\
\textbf{Input:} Locations $x_1$, $x_2$, ..., $x_n$ of gas stations along a highway.
\\ \\
\textbf{Output:} Stoppage time at each station which minimizes the total time which you are stopped for gas.
\\ \\
\textbf{Theorem:} The given "next-station" algorithm (NS) is correct / optimal.
\\ \\
\textbf{Proof:} Assume to reach a contradiction that there exists an input on which NS produces an unacceptable output.
\\ \\
Let $S_i$ be the stoppage time at each station
\\ \& NS(I)=\{$S_{\alpha(1)}$, $S_{\alpha(2)}$, ..., $S_{\alpha(n)}$\} be the output of NS on input I
\\ \& OPT(I)=\{$S_{\beta(1)}$, $S_{\beta(2)}$, ..., $S_{\beta(n)}$\} be the optimal solution that agrees with NS(I) the most number of steps
\\ \\ \\ \\ \\ \\
Let $x_k$ be the first stop of the trip upon which NS(I) differs from OPT(I)
\\ \\ By the definition of NS, filling up for any time less than $S_\alpha(k)$ would not get you to the next station and since $S_{\alpha(k)}$ $\neq$ $S_{\beta(k)}$, then we know that $S_{\beta(k)}$ $\textgreater$ $S_{\alpha(k)}$
\\ \\ Let time t be defined as the difference between those stoppage times, t = $S_{\beta(k)}$ - $S_{\alpha(k)}$
\\ \\ Therefore $S_{\beta(k)}$ = $S_{\alpha(k)}$ + t:
\\ \\ \\ \\ \\ \\ \\ \\
Let OPT'(I) = OPT(I) with time t shifted to the next stop: 
\\
\centerline{$S'_k$ = $S_k$ - t}
\centerline{$S'_{k+1}$ = $S_{k+1} + t$}
\\ \\ Clearly, we haven't changed the total stoppage time and we know that we can reach the next stop by the definition of NS. Furthermore we add the difference to the stop at $x_{k+1}$, so we will undoubtably be able to reach the station at $x_{k+2}$ -- and all future stations. 
\\ \\ $\therefore$ We have reached a contradiction because OPT' agrees with NS for one additional step despite OPT being defined as agreeing with NS for the most number of steps.

\end{homeworkProblem}

\pagebreak

\begin{homeworkProblem}
\textbf{(b)}
\\
\textbf{Input:} Locations $x_1$, ..., $x_n$ of gas stations along a highway.
\\ \\
\textbf{Output:} Stoppage time at each station which minimizes the total time which you are stopped for gas.
\\ \\
\textbf{Theorem:} The given "fill-up" algorithm (FU) is not correct or optimal.
\\ \\
\textbf{Proof:}
\\
\\ Consider the input I = $\{x_1, x_2\}$ where $x_2 - x_1$ = $\frac{1}{2} \frac{C}{F}$ 
\\ \\ Note that to reach $x_2$ from $x_1$ it would take $F(x_2 - x_1) = F(\frac{1}{2} \frac{C}{F}) = \frac{1}{2} C$ liters of gas
\\ \\ At $x_1$ the tank is filled to capacity according to FU, taking $\frac{C}{r}$ minutes. But it is possible and correct to reach $x_2$ using only $\frac{1}{2}C$ liters, making the optimal time $\frac{1}{2} \frac{C}{r}$ minutes.
\\ \\ $\therefore$ FU is not correct or optimal. 
\end{homeworkProblem}

\pagebreak

\begin{homeworkProblem}
\centerline{\textbf{Problem 7}}
\leavevmode
\\ \\
\textbf{Algorithm:} If a page is not in fast memory and an eviction must occur, evict the page that does not occur again or whose next use will occur farthest in the future.
\\ \\
\textbf{Input:} A sequence I = \{$P_1, P_2, ..., P_n$\} of page requests
\\ \\
\textbf{Output:} Minimum cardinality of a sequence of evicted pages E
\\ \\
\textbf{Theorem:} The given "most points" algorithm (MP) is incorrect.
\\ \\
\textbf{Proof:} 
\\ \\
Consider the input P = \{0.5, 1.0, 1.49, 1.51, 2.0, 2.5\}
\\ \\ \\ \\ \\ \\ 
\\ Clearly the interval $I_1$ = (1, 2) covers the most points in P, so MP adds $I_1$ to S.  Now MP has no choice but to add separate intervals $I_2$ = (0.5, 1.5) and $I_3$ = (1.5, 2.5) to cover the remaining two points (0.5 and 2.5, respectively). 
\\ \\ \centerline{\textbf{Thus MP has arrived at S = $\{ I_1,  I_2, I_3\}$}}
\\ \\ \\
Now consider S = \{(0.5, 1.5), (1.5, 2.5)\}
\\ \\ \\ \\ \\ \\ 
Clearly OPT(P) = S, $| OPT(P) | < | MP(P) |$, and MP is not optimal for P.
\\ \\ $\therefore$ MP is not correct.

\end{homeworkProblem}

\end{document}
