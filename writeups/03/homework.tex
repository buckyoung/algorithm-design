 
%
% Homework Details
%   - Title
%   - Due date
%   - Class
%   - Section/Time
%   - Instructor
%   - Author
%

\newcommand{\hmwkTitle}{Greedy Algorithms}
\newcommand{\hmwkProblems}{6 and 7}
\newcommand{\hmwkDueDate}{September 3, 2014}
\newcommand{\ClassName}{Algorithm Design}
\newcommand{\ClassNumber}{CS 1510}
\newcommand{\hmwkAuthorName}{Buck Young and Rob Brown}



%
% Basic Document Settings
%

\documentclass{article}

\usepackage{fancyhdr}
\usepackage{extramarks}
\usepackage{amsmath}
\usepackage{amsthm}
\usepackage{amsfonts}
\usepackage{tikz}
\usepackage[plain]{algorithm}
\usepackage{algpseudocode}
\usepackage{amssymb}

\usetikzlibrary{automata,positioning}


\topmargin=-0.45in
\evensidemargin=0in
\oddsidemargin=0in
\textwidth=6.5in
\textheight=9.0in
\headsep=0.25in\newcommand{\hmwkClassTime}{Section A}
\linespread{1.1}

\renewcommand\headrulewidth{0.4pt}
\renewcommand\footrulewidth{0.4pt}
\setlength\parindent{0pt}

%
% Create Problem Sections
%

\newcommand{\enterProblemHeader}[1]{
    \nobreak\extramarks{}{Problem \arabic{#1} continued on next page\ldots}\nobreak{}
    \nobreak\extramarks{Problem \arabic{#1} (continued)}{Problem \arabic{#1} continued on next page\ldots}\nobreak{}
}

\newcommand{\exitProblemHeader}[1]{
    \nobreak\extramarks{Problem \arabic{#1} (continued)}{Problem \arabic{#1} continued on next page\ldots}\nobreak{}
    \stepcounter{#1}
    \nobreak\extramarks{Problem \arabic{#1}}{}\nobreak{}
}

\setcounter{secnumdepth}{0}
\newcounter{partCounter}
\newcounter{homeworkProblemCounter}
\setcounter{homeworkProblemCounter}{1}
\nobreak\extramarks{Problem \arabic{homeworkProblemCounter}}{}\nobreak{}

\newenvironment{homeworkProblem}{
    \section{ }
    %\setcounter{partCounter}{1}
    %\enterProblemHeader{homeworkProblemCounter}
}{
    \exitProblemHeader{homeworkProblemCounter}
}



% 
% Header and Footer definition
%

\pagestyle{fancy}
\lhead{\ClassNumber\ - \ClassName}
\chead{\hmwkTitle}
\rhead{Problems \hmwkProblems}
\lfoot{\lastxmark}
\cfoot{\thepage}


%
% Title Page
%

\title{
    \vspace{2in}
	\textmd{\textbf{\ClassNumber}} \\
    \textmd{\textbf{\ClassName}} \\    
    \normalsize\vspace{0.1in}\small{\hmwkTitle} \\
    \normalsize\vspace{0.1in}\small{Problems \hmwkProblems} \\
	\normalsize\vspace{0.1in}\small{Due \hmwkDueDate}    \\
    \vspace{3in}
}

\author{\textbf{\hmwkAuthorName}}
\date{}

\renewcommand{\part}[1]{\textbf{\large Part \Alph{partCounter}}\stepcounter{partCounter}\\}






% 	% 	%	%	%	%	%
%	Document Start 		%
% 	% 	% 	% 	% 	% 	%

\begin{document}

\maketitle

\pagebreak




\begin{homeworkProblem}
\centerline{\textbf{Problem 6}}
\leavevmode
\\ \\
\textbf{(a)}
\\
\textbf{Input:} An ordered sequence of words $S = \{w_1, w_2, ... , w_n\}$ where the length of the $i^{th}$ word is $w_i$.
\\
\textbf{Output:} Total penalty P as described by the problem statement ($\Sigma P_i$ for each $P_i = K_i - L$) such that P is minimized. 
\\
\textbf{Theorem:} The given greedy algorithm G is correct.
\\ \\
\textbf{Proof:} 
Assume to reach a contradiction that there exists some input I on which G produces an unacceptable output. Let OPT be the optimal solution that agrees with G for the most number of steps. Let the first point of disagreement be the word $w_{L_k}$ on line $k$. The only way OPT can disagree with G is if OPT moves $w_{L_k}$ to a new line and G does not (obviously G will not fail to place a word on a line that it is able to).
\\ \\ \\ \\ \\ \\ \\ \\ \\ \\ \\ \\
Let OPT' be a modified version of OPT, such that the first word on each new line ($w_{L_k}, w_{L_{k+1}}, ..., w_{L_n}$) is moved to the previous line if possible. The logic governing this movement from the $i^{th}$ line to the $(i-1)^{th}$ line is as follows. 
\\ 
\begin{algorithmic}[1]
\For{$i \ge k$}
\If{$w_{L_i} \le (L - K'_{i-1})$}
	\If{$w_{L_i}$ is the ONLY word on the LAST line}
		\State remove the final line, as the word fits on the previous line
		\State $P'_i$ changed from $L - w_i$ to 0 
		\State $P'_{i-1} -= w_i$ 
		\State $\Delta P \le 0$
	\Else
		\State Move $w_i$ from $L_i$ to $L_{i-1}$ 
		\State $P_i += w_i$
		\State $P_{i-1} -= w_i$
		\State $\Delta P = 0$
	\EndIf
\Else
	\State Cannot move $w_{L_i}$ up
	\State $\Delta P = 0$
\EndIf 
\EndFor
\end{algorithmic}
\leavevmode
\\
Thus, any state that OPT' can reach will result in $\Delta P = P' - P \le 0$. This tells us that $OPT' \le OPT$, and we have that OPT' is correct. Additionally, we know that OPT' can fit $w_k$ on the prior line, since G made this selection. So for at least one word we are able to make OPT' more like G. Now OPT' is both correct and agrees with G for one step more than OPT. 
\\ \\
$\therefore$ OPT' contradicts the definition of OPT, and G is correct.

\end{homeworkProblem}

\pagebreak

\begin{homeworkProblem}
\textbf{(b)}

\end{homeworkProblem}

\pagebreak

\begin{homeworkProblem}
\centerline{\textbf{Problem 7}}
\leavevmode
\\ \\
\textbf{Algorithm:} If a page is not in fast memory and an eviction must occur, evict the page that does not occur again or whose next use will occur farthest in the future.
\\ \\
\textbf{Input:} A sequence I = \{$P_1, P_2, ..., P_n$\} of page requests
\\ \\
\textbf{Output:} Minimum cardinality of a sequence of evicted pages E
\\ \\
\textbf{Theorem:} The "farthest in the future" algorithm F is correct.
\\ \\
\textbf{Proof:} Assume to reach a contraction that there exists input I on which F produces unacceptable output.
\\ \\
Let F(I)=\{$P_{\alpha(1)}$, $P_{\alpha(2)}$, ..., $P_{\alpha(n)}$\} be the output of F on input I
\\ \& OPT(I)=\{$P_{\beta(1)}$, $P_{\beta(2)}$, ..., $P_{\beta(n)}$\} be the optimal solution that agrees with F(I) the most number of steps
\\ \\
Consider the first point of disagreement k between OPT and F
\\ \\ \\ \\ \\ \\ \\ \\ 
Let OPT' equal OPT with $P_{\alpha(k)}$ chosen at k instead of $P_{\beta(k)}$
\\ \\
Clearly OPT' is more similar to F (for one additional step)
\\ \\
To show that OPT' is still correct, consider the four possible cases for input I:
\\ \\
\centerline{Case 1}
\\ \\ \\ \\ 
In this case, neither $P_{\alpha(k)}$ nor $P_{\beta(k)}$ exist in the input after time-step k. Therefore $|E|$ and $|E'|$ do not change.
\\ \\
\centerline{Case 2}
\\ \\ \\ \\ 
In this case, only $P_{\beta(k)}$ exists in the input after time-step k. Since $P_{\beta(k)}$ is needed at some point in the future (and OPT has evicted it at time-step k), OPT is guaranteed to need at least one additional eviction. Therefore $|E|$ \textgreater $|E'|$.
\\ \\
\centerline{Case 3}
\\ \\ \\ \\ 
In this case, both $P_{\alpha(k)}$ and $P_{\beta(k)}$ exist in the input after time-step k. Therefore $|E|$ must account for an eviction to bring $P_{\beta(k)}$ back into fast memory and $|E'|$ must account for an eviction to bring $P_{\alpha(k)}$ back into fast memory.  
\\ \\
\centerline{Case 4}
\\ \\ \\ \\ 
In this case, only $P_{\alpha(k)}$ exists in the input after time-step k. By the definition of F, if $P_{\beta(k)}$ did not exist in the input after time-step k then F would have chosen it. Therefore this is an impossible state.
\\ \\
In all cases, $|E|$ $\geq$ $|E'|$. Therefore (since we are minimizing) OPT $\leq$ OPT'
\\ \\ $\therefore$ We have reached a contradiction because OPT' agrees with F for one additional step (and still produces a correct output) despite OPT being defined as agreeing with F for the most number of steps.
\end{homeworkProblem}

\end{document}
