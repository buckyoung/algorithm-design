 
%
% Homework Details
%   - Title
%   - Due date
%   - Class
%   - Section/Time
%   - Instructor
%   - Author
%


%
% Basic Document Settings
%

\documentclass{article}

\usepackage{fancyhdr}
\usepackage{extramarks}
\usepackage{amsmath}
\usepackage{amsthm}
\usepackage{amsfonts}
\usepackage{tikz}
\usepackage[plain]{algorithm}
\usepackage[noend]{algpseudocode}
\usepackage{amssymb}

\usetikzlibrary{automata,positioning}


\newcommand*\circled[1]{\tikz[baseline=(char.base)]{
            \node[shape=circle,draw,inner sep=2pt] (char) {#1};}}
            
\topmargin=-0.45in
\evensidemargin=0in
\oddsidemargin=0in
\textwidth=6.5in
\textheight=9.0in
\headsep=0.25in\newcommand{\hmwkClassTime}{Section A}
\linespread{1.1}

\renewcommand\headrulewidth{0.4pt}
\renewcommand\footrulewidth{0.4pt}
\setlength\parindent{0pt}

%
% Create Problem Sections
%

\newcommand{\enterProblemHeader}[1]{
    \nobreak\extramarks{}{Problem \arabic{#1} continued on next page\ldots}\nobreak{}
    \nobreak\extramarks{Problem \arabic{#1} (continued)}{Problem \arabic{#1} continued on next page\ldots}\nobreak{}
}

\newcommand{\exitProblemHeader}[1]{
    \nobreak\extramarks{Problem \arabic{#1} (continued)}{Problem \arabic{#1} continued on next page\ldots}\nobreak{}
    \stepcounter{#1}
    \nobreak\extramarks{Problem \arabic{#1}}{}\nobreak{}
}

\setcounter{secnumdepth}{0}
\newcounter{partCounter}
\newcounter{homeworkProblemCounter}
\setcounter{homeworkProblemCounter}{1}
\nobreak\extramarks{Problem \arabic{homeworkProblemCounter}}{}\nobreak{}

\newenvironment{homeworkProblem}{
    \section{ }
    %\setcounter{partCounter}{1}
    %\enterProblemHeader{homeworkProblemCounter}
}{
    \exitProblemHeader{homeworkProblemCounter}
}



% 
% Header and Footer definition
%

\pagestyle{fancy}
\lfoot{\lastxmark}
\cfoot{$_{Buck}$ $_{Young}$ $_{and}$ $_{Rob}$ $_{Brown}$}


%
% Title Page
%

\title{
    \vspace{2in}
	\textmd{\textbf{\ClassNumber}} \\
    \textmd{\textbf{\ClassName}} \\    
    \normalsize\vspace{0.1in}\small{\hmwkTitle} \\
    \normalsize\vspace{0.1in}\small{Problems \hmwkProblems} \\
	\normalsize\vspace{0.1in}\small{Due \hmwkDueDate}    \\
    \vspace{3in}
}

\author{\textbf{\hmwkAuthorName}}
\date{}

\renewcommand{\part}[1]{\textbf{\large Part \Alph{partCounter}}\stepcounter{partCounter}\\}






% 	% 	%	%	%	%	%
%	Document Start 		%
% 	% 	% 	% 	% 	% 	%

\begin{document}
\pagebreak

\begin{homeworkProblem}
\centerline{\textbf{Problem 1 - part 1}}
\leavevmode
(2 Points) Consider the problem of computing the AND of n bits.
\\ \\
\textbf{Input:} n bits
\\ \textbf{Output:} The AND of the bits
\\ \\ \centerline{\textbf{Give an algorithm that runs in time O(log n) using n processors on an EREW PRAM.}}
\\
\\Program AND( $b_1$...$b_n$, p ):
\begin{algorithmic}[1]
\State return AND( $b_1$...$b_{n/2}$, p/2 ) $\wedge$ AND( $b_{(n/2) + 1}$...$b_n$, p/2 )
\end{algorithmic}
\leavevmode
\\ \\ \\ \\ \\ \\ \\ \\
\\ \\ \centerline{\textbf{What is the efficiency of this algorithm?}}
\\ \\ \\ \\ \\ \\ \\ \\
\\ \\ \centerline{\textbf{Using the folding principle, what upper bound would you get for this algorithm on $n^{1/3}$ processors?}}
\end{homeworkProblem}
\pagebreak

\begin{homeworkProblem}
\centerline{\textbf{Problem 1 - part 2}}
\leavevmode
\\ Incomplete
\\\\
\centerline{\textbf{Problem 1 - part 3}}
\leavevmode
\\ \\ \centerline{\textbf{Give an algorithm that runs in time O(1) using n processors on a CRCW Common PRAM.}}
\\
\\ Algorithm for processor $P_i$:
\begin{algorithmic}[1]
\If{$b_i$ = 1}
	\State ANSWER = 1
\EndIf
\end{algorithmic}
\leavevmode
\\ \\ \\ \\ \\ \\ \\ \\
\\ \\ \centerline{\textbf{What is the efficiency of this algorithm?}}
\\ \\ \\ \\ \\ \\ \\ \\
\\ \\ \centerline{\textbf{Using the folding principle, what upper bound would you get for this algorithm on $n^{2/3}$ processors?}}
\end{homeworkProblem}
\pagebreak
\end{document}
