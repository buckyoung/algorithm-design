 
%
% Homework Details
%   - Title
%   - Due date
%   - Class
%   - Section/Time
%   - Instructor
%   - Author
%

\newcommand{\hmwkTitle}{Greedy Algorithms \& Dynamic Programming}
\newcommand{\hmwkProblems}{12, 13 and 1}
\newcommand{\hmwkDueDate}{Monday September 8, 2014}
\newcommand{\ClassName}{Algorithm Design}
\newcommand{\ClassNumber}{CS 1510}
\newcommand{\hmwkAuthorName}{Buck Young and Rob Brown}



%
% Basic Document Settings
%

\documentclass{article}

\usepackage{fancyhdr}
\usepackage{extramarks}
\usepackage{amsmath}
\usepackage{amsthm}
\usepackage{amsfonts}
\usepackage{tikz}
\usepackage[plain]{algorithm}
\usepackage[noend]{algpseudocode}
\usepackage{amssymb}

\usetikzlibrary{automata,positioning}


\topmargin=-0.45in
\evensidemargin=0in
\oddsidemargin=0in
\textwidth=6.5in
\textheight=9.0in
\headsep=0.25in\newcommand{\hmwkClassTime}{Section A}
\linespread{1.1}

\renewcommand\headrulewidth{0.4pt}
\renewcommand\footrulewidth{0.4pt}
\setlength\parindent{0pt}

%
% Create Problem Sections
%

\newcommand{\enterProblemHeader}[1]{
    \nobreak\extramarks{}{Problem \arabic{#1} continued on next page\ldots}\nobreak{}
    \nobreak\extramarks{Problem \arabic{#1} (continued)}{Problem \arabic{#1} continued on next page\ldots}\nobreak{}
}

\newcommand{\exitProblemHeader}[1]{
    \nobreak\extramarks{Problem \arabic{#1} (continued)}{Problem \arabic{#1} continued on next page\ldots}\nobreak{}
    \stepcounter{#1}
    \nobreak\extramarks{Problem \arabic{#1}}{}\nobreak{}
}

\setcounter{secnumdepth}{0}
\newcounter{partCounter}
\newcounter{homeworkProblemCounter}
\setcounter{homeworkProblemCounter}{1}
\nobreak\extramarks{Problem \arabic{homeworkProblemCounter}}{}\nobreak{}

\newenvironment{homeworkProblem}{
    \section{ }
    %\setcounter{partCounter}{1}
    %\enterProblemHeader{homeworkProblemCounter}
}{
    \exitProblemHeader{homeworkProblemCounter}
}



% 
% Header and Footer definition
%

\pagestyle{fancy}
\lhead{\ClassNumber\ - \ClassName}
\chead{\hmwkTitle}
\rhead{Problems \hmwkProblems}
\lfoot{\lastxmark}
\cfoot{\thepage}


%
% Title Page
%

\title{
    \vspace{2in}
	\textmd{\textbf{\ClassNumber}} \\
    \textmd{\textbf{\ClassName}} \\    
    \normalsize\vspace{0.1in}\small{\hmwkTitle} \\
    \normalsize\vspace{0.1in}\small{Problems \hmwkProblems} \\
	\normalsize\vspace{0.1in}\small{Due \hmwkDueDate}    \\
    \vspace{3in}
}

\author{\textbf{\hmwkAuthorName}}
\date{}

\renewcommand{\part}[1]{\textbf{\large Part \Alph{partCounter}}\stepcounter{partCounter}\\}






% 	% 	%	%	%	%	%
%	Document Start 		%
% 	% 	% 	% 	% 	% 	%

\begin{document}

\maketitle

\pagebreak




\begin{homeworkProblem}
\centerline{\textbf{Problem 12}}
\leavevmode
\textbf{Input:}
\\
\textbf{Output:} 
\\
\textbf{Theorem:} 
\\ \\
\textbf{Proof:} 
\end{homeworkProblem}



\pagebreak



\begin{homeworkProblem}
\centerline{\textbf{Problem 13}}
\leavevmode
\textbf{(a)} 
\\
\textbf{Input:}
\\
\textbf{Output:} 
\\ \\
\textbf{Algorithm:}
\\ \\
\textbf{Theorem:} 
\\
\textbf{Proof:} 
\end{homeworkProblem}



\pagebreak



\begin{homeworkProblem}
\centerline{\textbf{Problem 13}}
\leavevmode
\textbf{(b)} 
\\
\textbf{Input:}
\\
\textbf{Output:} 
\\ \\
\textbf{Algorithm:}
\\ \\
\textbf{Theorem:} 
\\
\textbf{Proof:} 
\end{homeworkProblem}



\pagebreak



\begin{homeworkProblem}
\centerline{\textbf{Problem 1}}
\leavevmode
\textbf{(a)} 
\\
\begin{algorithmic}
\Procedure{T}{int $n$}
	\If{n == 0 or n==1}
  		\State return 2;
	\EndIf
	
	\State SUM = 0
	\For {$1 \le i \le n-1$}
		\State SUM+=T(i)*T(i-1)
	\EndFor	
	\State \Return SUM
\EndProcedure
\end{algorithmic}
\leavevmode
\\
\textbf{(b)} 
\\
\begin{algorithmic}
\Procedure{T}{int $n$}
	\If{n == 0 or n==1}
  		\State return 2;
	\EndIf
	
	\State T[0] = 2, T[1] = 2
	\For {$2 \le i \le n$}
		\State SUM = 0
		\For {$1 \le j \le n-1$}
			\State SUM+=T[j]*T[j-1]
		\EndFor
	T[i] = SUM
	\EndFor	
	\State \Return T[n]
\EndProcedure
\end{algorithmic}
\leavevmode
\\
\textbf{(c)} 
\\
\begin{algorithmic}
\Procedure{T}{int $n$}
	\If{n == 0 or n==1}
  		\State return 2;
	\EndIf
	
	\State T[0] = 2, T[1] = 2, T[2] = 4
	\For {$3 \le i \le n$}
		\State T[i] = T[i-1]*T[i-2] + T[i-1]
	\EndFor	
	\State \Return T[n]
\EndProcedure
\end{algorithmic}
\leavevmode
\\

\end{homeworkProblem}



\end{document}