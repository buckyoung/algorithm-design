 
%
% Homework Details
%   - Title
%   - Due date
%   - Class
%   - Section/Time
%   - Instructor
%   - Author
%


%
% Basic Document Settings
%

\documentclass{article}

\usepackage{fancyhdr}
\usepackage{extramarks}
\usepackage{amsmath}
\usepackage{amsthm}
\usepackage{amsfonts}
\usepackage{tikz}
\usepackage[plain]{algorithm}
\usepackage[noend]{algpseudocode}
\usepackage{amssymb}

\usetikzlibrary{automata,positioning}


\topmargin=-0.45in
\evensidemargin=0in
\oddsidemargin=0in
\textwidth=6.5in
\textheight=9.0in
\headsep=0.25in\newcommand{\hmwkClassTime}{Section A}
\linespread{1.1}

\renewcommand\headrulewidth{0.4pt}
\renewcommand\footrulewidth{0.4pt}
\setlength\parindent{0pt}

%
% Create Problem Sections
%

\newcommand{\enterProblemHeader}[1]{
    \nobreak\extramarks{}{Problem \arabic{#1} continued on next page\ldots}\nobreak{}
    \nobreak\extramarks{Problem \arabic{#1} (continued)}{Problem \arabic{#1} continued on next page\ldots}\nobreak{}
}

\newcommand{\exitProblemHeader}[1]{
    \nobreak\extramarks{Problem \arabic{#1} (continued)}{Problem \arabic{#1} continued on next page\ldots}\nobreak{}
    \stepcounter{#1}
    \nobreak\extramarks{Problem \arabic{#1}}{}\nobreak{}
}

\setcounter{secnumdepth}{0}
\newcounter{partCounter}
\newcounter{homeworkProblemCounter}
\setcounter{homeworkProblemCounter}{1}
\nobreak\extramarks{Problem \arabic{homeworkProblemCounter}}{}\nobreak{}

\newenvironment{homeworkProblem}{
    \section{ }
    %\setcounter{partCounter}{1}
    %\enterProblemHeader{homeworkProblemCounter}
}{
    \exitProblemHeader{homeworkProblemCounter}
}



% 
% Header and Footer definition
%

\pagestyle{fancy}
\lfoot{\lastxmark}
\cfoot{$_{Buck}$ $_{Young}$ $_{and}$ $_{Rob}$ $_{Brown}$}


%
% Title Page
%

\title{
    \vspace{2in}
	\textmd{\textbf{\ClassNumber}} \\
    \textmd{\textbf{\ClassName}} \\    
    \normalsize\vspace{0.1in}\small{\hmwkTitle} \\
    \normalsize\vspace{0.1in}\small{Problems \hmwkProblems} \\
	\normalsize\vspace{0.1in}\small{Due \hmwkDueDate}    \\
    \vspace{3in}
}

\author{\textbf{\hmwkAuthorName}}
\date{}

\renewcommand{\part}[1]{\textbf{\large Part \Alph{partCounter}}\stepcounter{partCounter}\\}






% 	% 	%	%	%	%	%
%	Document Start 		%
% 	% 	% 	% 	% 	% 	%

\begin{document}
\pagebreak

\begin{homeworkProblem}
\centerline{\textbf{Problem 16}}
\leavevmode
\\
\textbf{Input:} Positive integers $v_1$, ..., $v_n$
\\ \\ \textbf{Output:}  A subset $S$ of the integers such that $\sum_{v_i \in S}v_i^3$ = $\prod_{v_i \in S} v_i$
\\ \\ \textbf{Algorithm:} The "Subset-Sum-Product" algorithm SSP is defined and detailed below.
\\ \\ \textbf{Pruning Rule:} If two nodes on the same level have the same sum ($\sum$) and same product ($\prod$), prune either.
\\ \\
\begin{algorithmic}[1]
\State SSP[ 0 ][ 0 ][ 0 ] = 1 //initialize 
\For{lvl = 0 to n}
	\For{S = 0 to L}
		\For{P = 0 to L}
			\If{SSP[ lvl ][ S ][ P ] is defined}
				\State SSP[ lvl + 1 ][ S ][ P ] = 1 //left-child
				\State SSP[ lvl + 1 ][ S + $v_{lvl + 1}^3$ ][ P * $v_{lvl +1}$ ] = 1 //right-child
			\EndIf 
		\EndFor
	\EndFor
\EndFor
\end{algorithmic}
\leavevmode
\\ \\
Our algorithm cycles through the levels from top to bottom and through all possible sums and products from left to right. By our pruning rule, if two configurations lead to the same sum and the same product on the same level then one is pruned (does not matter which). For the "left-child" in the tree, we are choosing to take the same value as the "parent". For the "right-child" in the tree, we are summing and multiplying in the value of that level according to the given expressions.
\\ \\
Ultimately, we will arrive at a solution when the sum and product indices are equal. At this point, we can backtrace through the array to construct the subset. The backtrace would include subtracting the cube of the value at this level from S and dividing P by the value at this level, and then going up by one level. 
\\ \\ \\ \\
The justification for the chosen pruning rule is that at any level, if the sum and the product are the same then anything we add or multiply to those values will be identical at the next level and in the future. Consider this tree as an example:
\end{homeworkProblem}
\pagebreak

\end{document}
