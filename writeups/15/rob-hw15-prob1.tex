 
%
% Homework Details
%   - Title
%   - Due date
%   - Class
%   - Section/Time
%   - Instructor
%   - Author
%


%
% Basic Document Settings
%

\documentclass{article}

\usepackage{fancyhdr}
\usepackage{extramarks}
\usepackage{amsmath}
\usepackage{amsthm}
\usepackage{amsfonts}
\usepackage{tikz}
\usepackage[plain]{algorithm}
\usepackage[noend]{algpseudocode}
\usepackage{amssymb}

\usetikzlibrary{automata,positioning}


\topmargin=-0.45in
\evensidemargin=0in
\oddsidemargin=0in
\textwidth=6.5in
\textheight=9.0in
\headsep=0.25in\newcommand{\hmwkClassTime}{Section A}
\linespread{1.1}

\renewcommand\headrulewidth{0.4pt}
\renewcommand\footrulewidth{0.4pt}
\setlength\parindent{0pt}

%
% Create Problem Sections
%

\newcommand{\enterProblemHeader}[1]{
    \nobreak\extramarks{}{Problem \arabic{#1} continued on next page\ldots}\nobreak{}
    \nobreak\extramarks{Problem \arabic{#1} (continued)}{Problem \arabic{#1} continued on next page\ldots}\nobreak{}
}

\newcommand{\exitProblemHeader}[1]{
    \nobreak\extramarks{Problem \arabic{#1} (continued)}{Problem \arabic{#1} continued on next page\ldots}\nobreak{}
    \stepcounter{#1}
    \nobreak\extramarks{Problem \arabic{#1}}{}\nobreak{}
}

\setcounter{secnumdepth}{0}
\newcounter{partCounter}
\newcounter{homeworkProblemCounter}
\setcounter{homeworkProblemCounter}{1}
\nobreak\extramarks{Problem \arabic{homeworkProblemCounter}}{}\nobreak{}

\newenvironment{homeworkProblem}{
    \section{ }
    %\setcounter{partCounter}{1}
    %\enterProblemHeader{homeworkProblemCounter}
}{
    \exitProblemHeader{homeworkProblemCounter}
}



% 
% Header and Footer definition
%

\pagestyle{fancy}
\lfoot{\lastxmark}
\cfoot{$_{Buck}$ $_{Young}$ $_{and}$ $_{Rob}$ $_{Brown}$}


%
% Title Page
%

\title{
    \vspace{2in}
	\textmd{\textbf{\ClassNumber}} \\
    \textmd{\textbf{\ClassName}} \\    
    \normalsize\vspace{0.1in}\small{\hmwkTitle} \\
    \normalsize\vspace{0.1in}\small{Problems \hmwkProblems} \\
	\normalsize\vspace{0.1in}\small{Due \hmwkDueDate}    \\
    \vspace{3in}
}

\author{\textbf{\hmwkAuthorName}}
\date{}

\renewcommand{\part}[1]{\textbf{\large Part \Alph{partCounter}}\stepcounter{partCounter}\\}






% 	% 	%	%	%	%	%
%	Document Start 		%
% 	% 	% 	% 	% 	% 	%

\begin{document}
\pagebreak

\begin{homeworkProblem}
\centerline{\textbf{Problem 1}} 
\leavevmode \\
Let A and B both be $n \times n$ matrices. We construct the following algorithm to multiply A and B by depending on the subproblem of multiplying two upper-triangular matrices ($function\;MP$).
\\
\begin{algorithmic}[1]
\Function{M}{A, B}
	\State $A_{triangular} =
		 	 \begin{bmatrix} 		0_{n \times n} 		& 0_{n \times n} 	& 0_{n \times n} \\ 
								 	0_{n \times n}		& 0_{n \times n} 	& 0_{n \times n} \\
								 	0_{n \times n} 		& A 				& 0_{n \times n}
			 \end{bmatrix}$
			 
	\State $B_{triangular} = 
		 	 \begin{bmatrix} 		0_{n \times n} 		& 0_{n \times n} 	& 0_{n \times n} \\ 
								 	B 					& 0_{n \times n} 	& 0_{n \times n} \\
								 	0_{n \times n} 		& 0_{n \times n} 	& 0_{n \times n}
			 \end{bmatrix}$
	\State $C = MT(A_{triangular}, B_{triangular}) = 
			 \begin{bmatrix} 		0_{n \times n} 		& 0_{n \times n} 	& 0_{n \times n} \\ 
								 	0_{n \times n}		& 0_{n \times n} 	& 0_{n \times n} \\
								 	AB					& 0_{n \times n} 	& 0_{n \times n}
			 \end{bmatrix}$
	\State \Return $C\,[\,0 \rightarrow n\,]\,[\,2n \rightarrow 3n\,]$
\EndFunction
\end{algorithmic}
\leavevmode
\\ \\
We construct the two upper triangular matrices $A_{triangular}$ and $B_{triangular}$ in $n \times n$ components. Note that each upper-triangular matrix is $3n \times 3n$, and thus has $9n^2$ items to fill. Clearly, this can be done for both matrices in $\Theta(n^2)$ time. 
\\ \\
We then calculate the product $C = A_{triangular} \times B_{triangular}$  according to some arbitrary algorithm for computing the product of two triangular matrices (ie, function MT) and return the "bottom left" sub-matrix of C, an $n \times n$ matrix. Clearly, iterating over and selecting all the elements in an $n \times n$ matrix takes $\Theta(n^2)$ time. Thus, all work "in addition" to the work of MP is $\Theta(n^2)$ time complexity.
\\ \\ 
If MT is then able to compute our matrix product in $\Theta(n^2)$ time, we will have an algorithm to multiply two arbitrary $n \times n$ matrices in $\Theta(n^2)$ time. 
\end{homeworkProblem}
\pagebreak

\end{document}
