 
%
% Homework Details
%   - Title
%   - Due date
%   - Class
%   - Section/Time
%   - Instructor
%   - Author
%


%
% Basic Document Settings
%

\documentclass{article}

\usepackage{fancyhdr}
\usepackage{extramarks}
\usepackage{amsmath}
\usepackage{amsthm}
\usepackage{amsfonts}
\usepackage{tikz}
\usepackage[plain]{algorithm}
\usepackage[noend]{algpseudocode}
\usepackage{amssymb}

\usetikzlibrary{automata,positioning}


\newcommand*\circled[1]{\tikz[baseline=(char.base)]{
            \node[shape=circle,draw,inner sep=2pt] (char) {#1};}}
            
\topmargin=-0.45in
\evensidemargin=0in
\oddsidemargin=0in
\textwidth=6.5in
\textheight=9.0in
\headsep=0.25in\newcommand{\hmwkClassTime}{Section A}
\linespread{1.1}

\renewcommand\headrulewidth{0.4pt}
\renewcommand\footrulewidth{0.4pt}
\setlength\parindent{0pt}

%
% Create Problem Sections
%

\newcommand{\enterProblemHeader}[1]{
    \nobreak\extramarks{}{Problem \arabic{#1} continued on next page\ldots}\nobreak{}
    \nobreak\extramarks{Problem \arabic{#1} (continued)}{Problem \arabic{#1} continued on next page\ldots}\nobreak{}
}

\newcommand{\exitProblemHeader}[1]{
    \nobreak\extramarks{Problem \arabic{#1} (continued)}{Problem \arabic{#1} continued on next page\ldots}\nobreak{}
    \stepcounter{#1}
    \nobreak\extramarks{Problem \arabic{#1}}{}\nobreak{}
}

\setcounter{secnumdepth}{0}
\newcounter{partCounter}
\newcounter{homeworkProblemCounter}
\setcounter{homeworkProblemCounter}{1}
\nobreak\extramarks{Problem \arabic{homeworkProblemCounter}}{}\nobreak{}

\newenvironment{homeworkProblem}{
    \section{ }
    %\setcounter{partCounter}{1}
    %\enterProblemHeader{homeworkProblemCounter}
}{
    \exitProblemHeader{homeworkProblemCounter}
}



% 
% Header and Footer definition
%

\pagestyle{fancy}
\lfoot{\lastxmark}
\cfoot{$_{Buck}$ $_{Young}$ $_{and}$ $_{Rob}$ $_{Brown}$}


%
% Title Page
%

\title{
    \vspace{2in}
	\textmd{\textbf{\ClassNumber}} \\
    \textmd{\textbf{\ClassName}} \\    
    \normalsize\vspace{0.1in}\small{\hmwkTitle} \\
    \normalsize\vspace{0.1in}\small{Problems \hmwkProblems} \\
	\normalsize\vspace{0.1in}\small{Due \hmwkDueDate}    \\
    \vspace{3in}
}

\author{\textbf{\hmwkAuthorName}}
\date{}

\renewcommand{\part}[1]{\textbf{\large Part \Alph{partCounter}}\stepcounter{partCounter}\\}






% 	% 	%	%	%	%	%
%	Document Start 		%
% 	% 	% 	% 	% 	% 	%

\begin{document}
\pagebreak

\begin{homeworkProblem}
\centerline{\textbf{Problem 10}}
\leavevmode
(2 points) Show that the Vertex Cover Problem is self-reducible. The decision problem is to take a
graph G and an integer k and decide if G has a vertex cover of size k or not. The optimization problem
takes a graph G, and returns a smallest vertex cover in G. So you must show that if the decision
problem has a polynomial time algorithm then the optimization problem also has a polynomial time algorithm. Recall that a vertex cover is a collection S of vertices with the property that every edge is
incident to a vertex in S.
\\ \\
\textbf{Input:} graph G
\\ \\ \textbf{Output:} the smallest Vertex Cover in G
\\ \\ \textbf{Theorem:} VC is self reducible 
\\ \\ \centerline{\textbf{optimization problem OPT $\leq$ decision problem DEC}}
\\ \\
Program OPT( G ):
\begin{algorithmic}[1]
\For{k = 1 to number of nodes in G}
	\If{ DEC( G, k ) }
		\State BREAK
	\EndIf
\EndFor
\\
\For{each node $x$}
	\State G' = G with x removed
	\If{ DEC( G', k-1 ) }
		\State result += x
		\State G = G'
		\State k = k - 1
	\EndIf
\EndFor
\end{algorithmic}
\leavevmode
\\ \\ This algorithm shows that if the decision problem has a polynomial time algorithm then the optimization problem does as well.
\\ \\This algorithm first finds the minimum size of the smallest Vertex Cover in G. Then it considers if each node $x$ should be in the resulting solution. It does this by removing x and checking if there is a VC of size k-1. If there is, we can safely add this node to the solution set. 
\end{homeworkProblem}
\pagebreak

\end{document}
