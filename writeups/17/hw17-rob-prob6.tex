 
%
% Homework Details
%   - Title
%   - Due date
%   - Class
%   - Section/Time
%   - Instructor
%   - Author
%


%
% Basic Document Settings
%

\documentclass{article}

\usepackage{fancyhdr}
\usepackage{extramarks}
\usepackage{amsmath}
\usepackage{amsthm}
\usepackage{amsfonts}
\usepackage{tikz}
\usepackage[plain]{algorithm}
\usepackage[noend]{algpseudocode}
\usepackage{amssymb}

\usetikzlibrary{automata,positioning}


\topmargin=-0.45in
\evensidemargin=0in
\oddsidemargin=0in
\textwidth=6.5in
\textheight=9.0in
\headsep=0.25in\newcommand{\hmwkClassTime}{Section A}
\linespread{1.1}

\renewcommand\headrulewidth{0.4pt}
\renewcommand\footrulewidth{0.4pt}
\setlength\parindent{0pt}

%
% Create Problem Sections
%

\newcommand{\enterProblemHeader}[1]{
    \nobreak\extramarks{}{Problem \arabic{#1} continued on next page\ldots}\nobreak{}
    \nobreak\extramarks{Problem \arabic{#1} (continued)}{Problem \arabic{#1} continued on next page\ldots}\nobreak{}
}

\newcommand{\exitProblemHeader}[1]{
    \nobreak\extramarks{Problem \arabic{#1} (continued)}{Problem \arabic{#1} continued on next page\ldots}\nobreak{}
    \stepcounter{#1}
    \nobreak\extramarks{Problem \arabic{#1}}{}\nobreak{}
}

\setcounter{secnumdepth}{0}
\newcounter{partCounter}
\newcounter{homeworkProblemCounter}
\setcounter{homeworkProblemCounter}{1}
\nobreak\extramarks{Problem \arabic{homeworkProblemCounter}}{}\nobreak{}

\newenvironment{homeworkProblem}{
    \section{ }
    %\setcounter{partCounter}{1}
    %\enterProblemHeader{homeworkProblemCounter}
}{
    \exitProblemHeader{homeworkProblemCounter}
}



% 
% Header and Footer definition
%

\pagestyle{fancy}
\lfoot{\lastxmark}
\cfoot{$_{Buck}$ $_{Young}$ $_{and}$ $_{Rob}$ $_{Brown}$}


%
% Title Page
%

\title{
    \vspace{2in}
	\textmd{\textbf{\ClassNumber}} \\
    \textmd{\textbf{\ClassName}} \\    
    \normalsize\vspace{0.1in}\small{\hmwkTitle} \\
    \normalsize\vspace{0.1in}\small{Problems \hmwkProblems} \\
	\normalsize\vspace{0.1in}\small{Due \hmwkDueDate}    \\
    \vspace{3in}
}

\author{\textbf{\hmwkAuthorName}}
\date{}

\renewcommand{\part}[1]{\textbf{\large Part \Alph{partCounter}}\stepcounter{partCounter}\\}






% 	% 	%	%	%	%	%
%	Document Start 		%
% 	% 	% 	% 	% 	% 	%

\begin{document}
\pagebreak

\begin{homeworkProblem}
\centerline{\textbf{Problem 6}}
\leavevmode
\\
\centerline{\textit{Show that if one of the following three problems has a polynomial time algorithm then they all do.}}\\ \\
\textbf{UISO:}  The input is two undirected graphs G and H. The problem is to determine if the graphs are isomorphic.
\\ \\ \textbf{DISO:}  The input is two directed graphs G and H. The problem is to determine if the graphs are isomorphic.
\\ \\ \textbf{DCOUNT:}  The input is two undirected graphs G and H, and an integer k. The problem is to determine if the graphs are isomorphic and all the vertices in each graph have degree k.
\\ \\ \\
\centerline{\textbf{UISO $\le$ DISO}}
\begin{algorithmic}[1]
\Function{UISO}{G H}
	\For{e=$v_i$ --- $v_j$ in G} 
		\State add $v_i \rightarrow v_j$ to G'
		\State add $v_j \rightarrow v_i$ to G'
	\EndFor
	\For{e=$v_i$ --- $v_j$ in H}
		\State add $v_i \rightarrow v_j$ to H'
		\State add $v_j \rightarrow v_i$ to H'
	\EndFor	
	\State \Return DISO(G', H')
\EndFunction
\end{algorithmic}
\leavevmode
\\ \\ \\
\centerline{\textbf{DISO $\le$ UISO}}
\begin{algorithmic}[1]
\Function{DISO}{G H}
	\State Let G' be a graph with no edges and twice as many verticies as G
	\State Let H' be a graph with no edges and twice as many verticies as H	
	\State//(One vertex for each in G, as well as a "NULL" vertex for each)
	\For{e=$[v_i \rightarrow v_j]$ in G} 
		\If{node G[j] has edge $[v_j \rightarrow v_i]$}
			\State add $[v_i$ --- $v_j]$ to G' 
		\Else
			\State add $[v_i$ --- $v_j]$ to G'\space\space\space\space\space\space\space\space\space\space //undirected edge for $[v_i \rightarrow v_j]$
			\State add $[v_j$ --- $v_{2|v| - i}]$ to G'\space\space\space //undirected edge for absence of $[v_j \rightarrow v_i]$
		\EndIf
	\EndFor
	\State \Return UISO(G', H')
\EndFunction
\pagebreak
\end{algorithmic}

\centerline{\textbf{UCOUNT $\le$ UISO}}
\begin{algorithmic}[1]
\Function{UISO}{G, H, k}
	\State $deg_k = True$
	\For{$v_i$ in G} 
		\If{$deg(v_i) \ne k$}
			\State $deg_k = False$
		\EndIf
	\EndFor
	\For{$v_i$ in H} 
		\If{$deg(v_i) \ne k$}
			\State $deg_k = False$
		\EndIf
	\EndFor	
	\State \Return UISO(G', H') \textbf{and} $deg_k$
\EndFunction
\end{algorithmic}
\leavevmode \\ \\
\centerline{\textbf{UISO $\le$ UCOUNT}}
\begin{algorithmic}[1]
\Function{UISO}{G, H}
	\State Let G' = G and H' = H
	\State Let $k$ be the max degree between H and G
	\For{$v_i$ in G} 
		\For{n=deg($v_i$) to $k$}  \space\space\space\space //doesnt enter if deg($v_i$) = k
			\State add $k+1$ nodes to G' and connect them as described below
			\State connect $v_i$ to the one remaining node with degree $k-1$
		\EndFor
	\EndFor
	\For{$v_i$ in H} 
		\For{n=deg($v_i$) to $k$} \space\space\space\space //doesnt enter if deg($v_i$) = k
			\State add $k+1$ nodes to H' and connect them as described below
			\State connect $v_i$ to the one remaining node with degree $k-1$
		\EndFor
	\EndFor	
	\State \Return UCOUNT(G', H', k)
\EndFunction
\end{algorithmic}
\leavevmode \\
To create a structure of nodes such that we can add one connection and still satisfy the condition that $degree = k$ for all nodes, we can add $k+1$ vertices and make every possible connection between these $k+1$ nodes. Every vertex in this graph has degree $k$. Now remove an edge between arbitrary vertices $v_i$ and $v_j$. Connect $v_i$ to itself, bringing its degree back to k. Now one edge can be added to $v_j$ and all degree conditions will be satisfied. See a few examples below. This exploitation is used to bring the degree of every node in H and G up to k.
\\ \\ Note that we can construct this special tree in polynomial time, so this algorithm is a poly-time reduction (as is the case with the others, which are are polynomial reductions by trivial inspection). 

\end{homeworkProblem}
\pagebreak

\end{document}
