 
%
% Homework Details
%   - Title
%   - Due date
%   - Class
%   - Section/Time
%   - Instructor
%   - Author
%

\newcommand{\hmwkTitle}{Greedy Algorithms}
\newcommand{\hmwkProblems}{9 and 10}
\newcommand{\hmwkDueDate}{September 5, 2014}
\newcommand{\ClassName}{Algorithm Design}
\newcommand{\ClassNumber}{CS 1510}
\newcommand{\hmwkAuthorName}{Buck Young and Rob Brown}



%
% Basic Document Settings
%

\documentclass{article}

\usepackage{fancyhdr}
\usepackage{extramarks}
\usepackage{amsmath}
\usepackage{amsthm}
\usepackage{amsfonts}
\usepackage{tikz}
\usepackage[plain]{algorithm}
\usepackage{algpseudocode}
\usepackage{amssymb}

\usetikzlibrary{automata,positioning}


\topmargin=-0.45in
\evensidemargin=0in
\oddsidemargin=0in
\textwidth=6.5in
\textheight=9.0in
\headsep=0.25in\newcommand{\hmwkClassTime}{Section A}
\linespread{1.1}

\renewcommand\headrulewidth{0.4pt}
\renewcommand\footrulewidth{0.4pt}
\setlength\parindent{0pt}

%
% Create Problem Sections
%

\newcommand{\enterProblemHeader}[1]{
    \nobreak\extramarks{}{Problem \arabic{#1} continued on next page\ldots}\nobreak{}
    \nobreak\extramarks{Problem \arabic{#1} (continued)}{Problem \arabic{#1} continued on next page\ldots}\nobreak{}
}

\newcommand{\exitProblemHeader}[1]{
    \nobreak\extramarks{Problem \arabic{#1} (continued)}{Problem \arabic{#1} continued on next page\ldots}\nobreak{}
    \stepcounter{#1}
    \nobreak\extramarks{Problem \arabic{#1}}{}\nobreak{}
}

\setcounter{secnumdepth}{0}
\newcounter{partCounter}
\newcounter{homeworkProblemCounter}
\setcounter{homeworkProblemCounter}{1}
\nobreak\extramarks{Problem \arabic{homeworkProblemCounter}}{}\nobreak{}

\newenvironment{homeworkProblem}{
    \section{ }
    %\setcounter{partCounter}{1}
    %\enterProblemHeader{homeworkProblemCounter}
}{
    \exitProblemHeader{homeworkProblemCounter}
}



% 
% Header and Footer definition
%

\pagestyle{fancy}
\lhead{\ClassNumber\ - \ClassName}
\chead{\hmwkTitle}
\rhead{Problems \hmwkProblems}
\lfoot{\lastxmark}
\cfoot{\thepage}


%
% Title Page
%

\title{
    \vspace{2in}
	\textmd{\textbf{\ClassNumber}} \\
    \textmd{\textbf{\ClassName}} \\    
    \normalsize\vspace{0.1in}\small{\hmwkTitle} \\
    \normalsize\vspace{0.1in}\small{Problems \hmwkProblems} \\
	\normalsize\vspace{0.1in}\small{Due \hmwkDueDate}    \\
    \vspace{3in}
}

\author{\textbf{\hmwkAuthorName}}
\date{}

\renewcommand{\part}[1]{\textbf{\large Part \Alph{partCounter}}\stepcounter{partCounter}\\}






% 	% 	%	%	%	%	%
%	Document Start 		%
% 	% 	% 	% 	% 	% 	%

\begin{document}

\maketitle

\pagebreak




\begin{homeworkProblem}
\centerline{\textbf{Problem 9}}
\leavevmode
\\ \\
\textbf{(a)}
\\
\textbf{Input:} A number n of skiers with heights $p_1, ... , p_n$, and n skies with heights $s_1, ... , s_n$.
\\
\textbf{Output:} The minimal average difference between skier height $p_i$ and ski height $s_{\alpha(i)}$: $\frac{1}{n} \sum\limits_{i=1}^n |p_i - s_{\alpha(i)}|$
\\
\textbf{Theorem:} The given "minimized height difference first" algorithm MF is correct.
\\ \\
\textbf{Proof:} Consider a "counter-example" algorithm CE to prove MF is incorrect.
\\ \\ Let $T$ be the total height difference (and $\frac{T}{n}$ be the average total height difference),
\\ \& $H_i$ be the height difference between a person $p_i$ and a chosen ski $s_{\alpha(i)}$.
\\ \\ \\ \\ \\ \\ \\ \\ \\ \\ \\ \\ \\ \\ \\ \\ \\ \\ \\ \\ \\ \\ \\ \\ 
Clearly $\frac{T_{MF}}{n}$ $\textgreater$ $\frac{T_{CE}}{n}$ and the problem wants us to minimize $\frac{T}{n}$. 
\\ \\ 
Also -- let OPT be the optimal solution to this problem -- we see that MF $\textless$ CE $\leq$ OPT.
\\ \\
$\therefore$ The given algorithm MF is sub-optimal and not correct by way of this counter-example.
\end{homeworkProblem}

\pagebreak

\begin{homeworkProblem}
\textbf{(b)} 
\\
\textbf{Input:} A number n of skiers with heights $p_1, ... , p_n$, and n skies with heights $s_1, ... , s_n$.
\\
\textbf{Output:} The minimal average difference between skier height $p_i$ and ski height $s_{\alpha(i)}$: $\frac{1}{n} \sum\limits_{i=1}^n |p_i - s_{\alpha(i)}|$
\\
\textbf{Theorem:} The given "shortest-skier to shortest-ski" algorithm SS is correct.
\\ \\
\textbf{Proof:} EXCHANGE ARGUMENT
\end{homeworkProblem}

\pagebreak

\begin{homeworkProblem}
\centerline{\textbf{Problem 10}}
\leavevmode
\\ \\
\textbf{(SJF)}
\\
\textbf{Input:} A collection of jobs $J_1, ... , J_n$, where the $i$th job is a tuple ($r_i, x_i$) of non-negative integers specifying the release time $r$ and size of the job $x$.
\\
\textbf{Output:} A preemptive feasible schedule on one processor that minimizes the total completion time $\sum\limits_{i=1}^n C_i$
\\
\textbf{Theorem:} That the given "shortest job first" algorithm SJF is correct.
\\ \\
\textbf{Proof:}  Consider a "counter-example" algorithm CE to prove SJF is incorrect.
\\ \\ Let $C$ be the total completion time and $C_i$ be the completion time for job $i$.
\\ \\ \\ \\ \\ \\ \\ \\ \\ \\ \\ \\ \\ \\ \\ \\ \\ \\ \\ \\ \\ \\ \\ \\ 
Clearly $C_{SJF}$ $\textgreater$ $C_{CE}$ and the problem requires us to minimize $C$.
\\ \\ 
Also -- let OPT be the optimal solution to this problem -- we see that SJF $\textless$ CE $\leq$ OPT.
\\ \\
$\therefore$ The given algorithm SJF is sub-optimal and not correct by way of this counter-example.
\end{homeworkProblem}

\pagebreak

\begin{homeworkProblem}
\textbf{(SRPT)}
\\
\textbf{Input:} A collection of jobs $J_1, ... , J_n$, where the $i$th job is a tuple ($r_i, x_i$) of non-negative integers specifying the release time $r$ and size of the job $x$.
\\
\textbf{Output:} A preemptive feasible schedule on one processor that minimizes the total completion time $\sum\limits_{i=1}^n C_i$
\\
\textbf{Theorem:} That the given "shortest remaining processing time" algorithm SRPT is correct.
\\ \\
\textbf{Proof:} EXCHANGE ARGUMENT
\end{homeworkProblem}

\end{document}
