 
%
% Homework Details
%   - Title
%   - Due date
%   - Class
%   - Section/Time
%   - Instructor
%   - Author
%


%
% Basic Document Settings
%

\documentclass{article}

\usepackage{fancyhdr}
\usepackage{extramarks}
\usepackage{amsmath}
\usepackage{amsthm}
\usepackage{amsfonts}
\usepackage{tikz}
\usepackage[plain]{algorithm}
\usepackage[noend]{algpseudocode}
\usepackage{amssymb}

\usetikzlibrary{automata,positioning}


\newcommand*\circled[1]{\tikz[baseline=(char.base)]{
            \node[shape=circle,draw,inner sep=2pt] (char) {#1};}}
            
\topmargin=-0.45in
\evensidemargin=0in
\oddsidemargin=0in
\textwidth=6.5in
\textheight=9.0in
\headsep=0.25in\newcommand{\hmwkClassTime}{Section A}
\linespread{1.1}

\renewcommand\headrulewidth{0.4pt}
\renewcommand\footrulewidth{0.4pt}
\setlength\parindent{0pt}

%
% Create Problem Sections
%

\newcommand{\enterProblemHeader}[1]{
    \nobreak\extramarks{}{Problem \arabic{#1} continued on next page\ldots}\nobreak{}
    \nobreak\extramarks{Problem \arabic{#1} (continued)}{Problem \arabic{#1} continued on next page\ldots}\nobreak{}
}

\newcommand{\exitProblemHeader}[1]{
    \nobreak\extramarks{Problem \arabic{#1} (continued)}{Problem \arabic{#1} continued on next page\ldots}\nobreak{}
    \stepcounter{#1}
    \nobreak\extramarks{Problem \arabic{#1}}{}\nobreak{}
}

\setcounter{secnumdepth}{0}
\newcounter{partCounter}
\newcounter{homeworkProblemCounter}
\setcounter{homeworkProblemCounter}{1}
\nobreak\extramarks{Problem \arabic{homeworkProblemCounter}}{}\nobreak{}

\newenvironment{homeworkProblem}{
    \section{ }
    %\setcounter{partCounter}{1}
    %\enterProblemHeader{homeworkProblemCounter}
}{
    \exitProblemHeader{homeworkProblemCounter}
}



% 
% Header and Footer definition
%

\pagestyle{fancy}
\lfoot{\lastxmark}
\cfoot{$_{Buck}$ $_{Young}$ $_{and}$ $_{Rob}$ $_{Brown}$}


%
% Title Page
%

\title{
    \vspace{2in}
	\textmd{\textbf{\ClassNumber}} \\
    \textmd{\textbf{\ClassName}} \\    
    \normalsize\vspace{0.1in}\small{\hmwkTitle} \\
    \normalsize\vspace{0.1in}\small{Problems \hmwkProblems} \\
	\normalsize\vspace{0.1in}\small{Due \hmwkDueDate}    \\
    \vspace{3in}
}

\author{\textbf{\hmwkAuthorName}}
\date{}

\renewcommand{\part}[1]{\textbf{\large Part \Alph{partCounter}}\stepcounter{partCounter}\\}






% 	% 	%	%	%	%	%
%	Document Start 		%
% 	% 	% 	% 	% 	% 	%

\begin{document}
\pagebreak

\begin{homeworkProblem}
\centerline{\textbf{Problem 3}}
\leavevmode
(2 points) Consider the problem of taking as input an integer n and an integer x, and creating an array A of n integers, where each entry of A is equal to x.
\\ \\ \textbf{Give an algorithm that runs in time O(log n) using n processors on an EREW PRAM.}
\\ \\ Algorithm INIT( a ... b, x, p ):
\begin{algorithmic}[1]
\If{p = 1}
	\For{each index $i$ from a ... b}
		\State A[ $i$ ] = x
	\EndFor
\Else
\State INIT( a ... b/2, x, p/2 )
\State INIT( (b/2)+1 ... b, x, p/2 )
\EndIf
\end{algorithmic}
\leavevmode
\\ \textbf{PRIMING CALL:} INIT( 1 ... n, x, p)
\\
\\ The above algorithm will split the indices of A into two lists (an upper and lower half) and then assign x to each entry in A. The base case is when the processor list is equal to 1 and the priming call is from 1 to n. This algorithm will have lg(p) levels and p leaves in the recursive call tree. As such:
\\ T(n, p) = n/p + lg(p)
\\and T(n, p=n) = 1 + lg(p) 
\\
\\- What is the efficiency of this algorithm?
\\
\\ E(n, p) = n / ( p*( (n/p) + lg(p) ) ) = n / ( n + p*lg(p) )
\\ and E(n, p=n) = 1 / lg(n)
\\
\\- Using the folding principle, what upper bound would you get on the running time for this
algorithm on n$^{1/3}$ processors?
\\
\\T(n, p=n$^{1/3}$) $\leq$ (1/n$^{2/3}$)(1 + lg(p))
\\ \\ \\ \\
\textbf{Give an algorithm that runs in time O(log n) using n/ log n processors on an EREW PRAM.}
\\ \\The algorithm is the same as above.
\\
\\- What is the efficiency of this algorithm?
\\
\\E(n, p=n/lg(n)) = 1
\\
\\- Using the folding principle, what upper bound would you get on the running time for this
algorithm on n$^{1/3}$ processors?
\\
\\T(n, p=n$^{1/3}$) $\leq$ (1/n$^{2/3}$)(1 + lg(p))
\\ \\ \\ \\
\textbf{Give an algorithm that runs in time O(1) using n processors on a CRCW Common PRAM.}
\\
\\ Algorithm for processor P$_i$ where i $\textless$ n: 
\\ $_{1:}$ $\ \ \ \ \ $A[ i ] = x
\\ \\ This algorithm will require CR of x and EW of each entry in A, using n processors. Also it will run in O(1) when p=n. Note the running times:
\\ T(n, p=n) = 1
\\ and T(n, p) = n/p
\\ \\- What is the efficiency of this algorithm?
\\
\\ E(n, p) = n / (p * n/p) = n/n = 1
\\ \\-Using the folding principle, what upper bound would you get on the running time for this
algorithm on n$^{2/3}$ processors?
\\ \\ T(n, p=n$^{2/3}$) $\leq$ (1/n$^{1/3}$)(1)
\end{homeworkProblem}
\pagebreak

\end{document}
