 
%
% Homework Details
%   - Title
%   - Due date
%   - Class
%   - Section/Time
%   - Instructor
%   - Author
%


%
% Basic Document Settings
%

\documentclass{article}

\usepackage{fancyhdr}
\usepackage{extramarks}
\usepackage{amsmath}
\usepackage{amsthm}
\usepackage{amsfonts}
\usepackage{tikz}
\usepackage[plain]{algorithm}
\usepackage[noend]{algpseudocode}
\usepackage{amssymb}

\usetikzlibrary{automata,positioning}


\newcommand*\circled[1]{\tikz[baseline=(char.base)]{
            \node[shape=circle,draw,inner sep=2pt] (char) {#1};}}
            
\topmargin=-0.45in
\evensidemargin=0in
\oddsidemargin=0in
\textwidth=6.5in
\textheight=9.0in
\headsep=0.25in\newcommand{\hmwkClassTime}{Section A}
\linespread{1.1}

\renewcommand\headrulewidth{0.4pt}
\renewcommand\footrulewidth{0.4pt}
\setlength\parindent{0pt}

%
% Create Problem Sections
%

\newcommand{\enterProblemHeader}[1]{
    \nobreak\extramarks{}{Problem \arabic{#1} continued on next page\ldots}\nobreak{}
    \nobreak\extramarks{Problem \arabic{#1} (continued)}{Problem \arabic{#1} continued on next page\ldots}\nobreak{}
}

\newcommand{\exitProblemHeader}[1]{
    \nobreak\extramarks{Problem \arabic{#1} (continued)}{Problem \arabic{#1} continued on next page\ldots}\nobreak{}
    \stepcounter{#1}
    \nobreak\extramarks{Problem \arabic{#1}}{}\nobreak{}
}

\setcounter{secnumdepth}{0}
\newcounter{partCounter}
\newcounter{homeworkProblemCounter}
\setcounter{homeworkProblemCounter}{1}
\nobreak\extramarks{Problem \arabic{homeworkProblemCounter}}{}\nobreak{}

\newenvironment{homeworkProblem}{
    \section{ }
    %\setcounter{partCounter}{1}
    %\enterProblemHeader{homeworkProblemCounter}
}{
    \exitProblemHeader{homeworkProblemCounter}
}



% 
% Header and Footer definition
%

\pagestyle{fancy}
\lfoot{\lastxmark}
\cfoot{$_{Buck}$ $_{Young}$ $_{and}$ $_{Rob}$ $_{Brown}$}


%
% Title Page
%

\title{
    \vspace{2in}
	\textmd{\textbf{\ClassNumber}} \\
    \textmd{\textbf{\ClassName}} \\    
    \normalsize\vspace{0.1in}\small{\hmwkTitle} \\
    \normalsize\vspace{0.1in}\small{Problems \hmwkProblems} \\
	\normalsize\vspace{0.1in}\small{Due \hmwkDueDate}    \\
    \vspace{3in}
}

\author{\textbf{\hmwkAuthorName}}
\date{}

\renewcommand{\part}[1]{\textbf{\large Part \Alph{partCounter}}\stepcounter{partCounter}\\}






% 	% 	%	%	%	%	%
%	Document Start 		%
% 	% 	% 	% 	% 	% 	%

\begin{document}
\pagebreak

\begin{homeworkProblem}
\centerline{\textbf{Problem 2}}
\leavevmode
( 2 points) You know that lots of famous computer scientists have tried to find a fast efficient parallel
algorithm for the following Boolean Formula Value Problem:
\\INPUT: A Boolean formula F and a truth assignment A of the variables in F. 
\\OUTPUT: 1 if A makes F true, and 0 otherwise.
\\ \\Moreover, most computer scientists believe that there is no fast efficient parallel algorithm for the Boolean Value Problem. You want to find a fast efficient parallel algorithm for some new problem N. After much effort you can not find a fast efficient parallel algorithm for N, nor a proof that N does not have a fast efficient parallel algorithm. How could you give evidence that finding a fast efficient parallel algorithm for N is at least a hard of a problem as finding a fast efficient parallel algorithm for Boolean Formula Value problem? Be as specific as possible, and explain how convincing the evidence is.
\\Note that "fast and efficient" means poly-log time with a polynomial number of processors. The term "poly-log" means bounded by O(log$^k$ n) for some constant k.
\\ \\ \\ \\
The best way to prove this would be to reduce the Boolean Formula Value Problem (BFVP) into our new problem N using poly-log (or less) time complexity. We could do the reduction by mapping the input of BFVP to the input for N and then running the algorithm N and translating the output. Therefore, if our new problem N had a fast and efficient parallel algorithm then so too would BFVP. Yet since BFVP does not have a fast and efficient parallel algorithm (as one has not been discovered by the greats), N must not either. 
\end{homeworkProblem}
\pagebreak

\end{document}
