 
%
% Homework Details
%   - Title
%   - Due date
%   - Class
%   - Section/Time
%   - Instructor
%   - Author
%

\newcommand{\hmwkTitle}{Dynamic Programming}
\newcommand{\hmwkProblems}{4 and 5}
\newcommand{\hmwkDueDate}{Friday September 12, 2014}
\newcommand{\ClassName}{Algorithm Design}
\newcommand{\ClassNumber}{CS 1510}
\newcommand{\hmwkAuthorName}{Buck Young and Rob Brown}



%
% Basic Document Settings
%

\documentclass{article}

\usepackage{fancyhdr}
\usepackage{extramarks}
\usepackage{amsmath}
\usepackage{amsthm}
\usepackage{amsfonts}
\usepackage{tikz}
\usepackage[plain]{algorithm}
\usepackage{algpseudocode}
\usepackage{amssymb}

\usetikzlibrary{automata,positioning}


\topmargin=-0.45in
\evensidemargin=0in
\oddsidemargin=0in
\textwidth=6.5in
\textheight=9.0in
\headsep=0.25in\newcommand{\hmwkClassTime}{Section A}
\linespread{1.1}

\renewcommand\headrulewidth{0.4pt}
\renewcommand\footrulewidth{0.4pt}
\setlength\parindent{0pt}

%
% Create Problem Sections
%

\newcommand{\enterProblemHeader}[1]{
    \nobreak\extramarks{}{Problem \arabic{#1} continued on next page\ldots}\nobreak{}
    \nobreak\extramarks{Problem \arabic{#1} (continued)}{Problem \arabic{#1} continued on next page\ldots}\nobreak{}
}

\newcommand{\exitProblemHeader}[1]{
    \nobreak\extramarks{Problem \arabic{#1} (continued)}{Problem \arabic{#1} continued on next page\ldots}\nobreak{}
    \stepcounter{#1}
    \nobreak\extramarks{Problem \arabic{#1}}{}\nobreak{}
}

\setcounter{secnumdepth}{0}
\newcounter{partCounter}
\newcounter{homeworkProblemCounter}
\setcounter{homeworkProblemCounter}{1}
\nobreak\extramarks{Problem \arabic{homeworkProblemCounter}}{}\nobreak{}

\newenvironment{homeworkProblem}{
    \section{ }
    %\setcounter{partCounter}{1}
    %\enterProblemHeader{homeworkProblemCounter}
}{
    \exitProblemHeader{homeworkProblemCounter}
}



% 
% Header and Footer definition
%

\pagestyle{fancy}
\lhead{\ClassNumber\ - \ClassName}
\chead{\hmwkTitle}
\rhead{Problems \hmwkProblems}
\lfoot{\lastxmark}
\cfoot{$_{Buck}$ $_{Young}$ $_{and}$ $_{Rob}$ $_{Brown}$}


%
% Title Page
%

\title{
    \vspace{2in}
	\textmd{\textbf{\ClassNumber}} \\
    \textmd{\textbf{\ClassName}} \\    
    \normalsize\vspace{0.1in}\small{\hmwkTitle} \\
    \normalsize\vspace{0.1in}\small{Problems \hmwkProblems} \\
	\normalsize\vspace{0.1in}\small{Due \hmwkDueDate}    \\
    \vspace{3in}
}

\author{\textbf{\hmwkAuthorName}}
\date{}

\renewcommand{\part}[1]{\textbf{\large Part \Alph{partCounter}}\stepcounter{partCounter}\\}






% 	% 	%	%	%	%	%
%	Document Start 		%
% 	% 	% 	% 	% 	% 	%

\begin{document}

\maketitle

\pagebreak




\begin{homeworkProblem}
\centerline{\textbf{Problem 4}}
\leavevmode \\
\textbf{Input: Two strings, $A=a_1a_2...a_m$ and $B=b_1b_2...b_n$}
\\ \\
\textbf{Output: The minimum cost steps to convert A to B according (where the cost of deletion is 3, the cost of insertion is 4, and the cost of replacement is 5.} 
\\ \\
\textbf{Algorithm:} 
\\
\begin{algorithmic}
	\State //set boundries (ie, base cases from recursion)
	\State $DIFF[m+1][n+1]$
	\For {$0 \le a \le m$}
		\State $DIFF[a][0]=3*a$ //if we get here, the "recersion" is done and we have no option but deletion
	\EndFor	
	\For {$0 \le b \le n$}
		\State $DIFF[0][b]=4*a$  //if we get here, the "recursion" is done and we have no option but insertion
	\EndFor	
	\\ \State //iterate in place of recursive call
	\For {$1 \le a \le m$}
		\For {$1 \le b \le n$}
			\If {A[a] == B[b]}
				\State $DIFF[a][b]= DIFF[a-1][b-1]+0$ //characters match, do nothing
			\Else
				\State $DIFF[a][b] = min(DIFF[a-1][b]+3, DIFF[a][b-1]+4, DIFF[a-1][b-1]+5)$
			\EndIf
		\EndFor
	\EndFor
\\ \\ \\
\begin{table}[h]
\begin{tabular}{ccccccccc}
  &                        &                         & a                       & b                       & c                       & a                       & b                       & c                       \\
  &                        & 0                       & 1                       & 2                       & 3                       & 4                       & 5                       & 6                       \\ \cline{3-9} 
  & \multicolumn{1}{c|}{0} & \multicolumn{1}{c|}{0}  & \multicolumn{1}{c|}{3}  & \multicolumn{1}{c|}{6}  & \multicolumn{1}{c|}{9}  & \multicolumn{1}{c|}{12} & \multicolumn{1}{c|}{15} & \multicolumn{1}{c|}{18} \\ \cline{3-9} 
a & \multicolumn{1}{c|}{1} & \multicolumn{1}{c|}{4}  & \multicolumn{1}{c|}{0}  & \multicolumn{1}{c|}{3}  & \multicolumn{1}{c|}{6}  & \multicolumn{1}{c|}{9}  & \multicolumn{1}{c|}{12} & \multicolumn{1}{c|}{15} \\ \cline{3-9} 
b & \multicolumn{1}{c|}{2} & \multicolumn{1}{c|}{8}  & \multicolumn{1}{c|}{4}  & \multicolumn{1}{c|}{0}  & \multicolumn{1}{c|}{3}  & \multicolumn{1}{c|}{6}  & \multicolumn{1}{c|}{9}  & \multicolumn{1}{c|}{12} \\ \cline{3-9} 
a & \multicolumn{1}{c|}{3} & \multicolumn{1}{c|}{12} & \multicolumn{1}{c|}{8}  & \multicolumn{1}{c|}{4}  & \multicolumn{1}{c|}{5}  & \multicolumn{1}{c|}{3}  & \multicolumn{1}{c|}{6}  & \multicolumn{1}{c|}{9}  \\ \cline{3-9} 
c & \multicolumn{1}{c|}{4} & \multicolumn{1}{c|}{16} & \multicolumn{1}{c|}{12} & \multicolumn{1}{c|}{8}  & \multicolumn{1}{c|}{4}  & \multicolumn{1}{c|}{7}  & \multicolumn{1}{c|}{10} & \multicolumn{1}{c|}{6}  \\ \cline{3-9} 
a & \multicolumn{1}{c|}{5} & \multicolumn{1}{c|}{20} & \multicolumn{1}{c|}{16} & \multicolumn{1}{c|}{12} & \multicolumn{1}{c|}{8}  & \multicolumn{1}{c|}{4}  & \multicolumn{1}{c|}{7}  & \multicolumn{1}{c|}{10} \\ \cline{3-9} 
b & \multicolumn{1}{c|}{6} & \multicolumn{1}{c|}{24} & \multicolumn{1}{c|}{20} & \multicolumn{1}{c|}{16} & \multicolumn{1}{c|}{12} & \multicolumn{1}{c|}{8}  & \multicolumn{1}{c|}{4}  & \multicolumn{1}{c|}{7}  \\ \cline{3-9} 
\end{tabular}
\end{table}
	
	
\end{algorithmic}
\end{homeworkProblem}



\pagebreak



\begin{homeworkProblem}
\centerline{\textbf{Problem 5}}
\leavevmode
\\ \\
Here are the generated tables for the given problem. Scratch-work for the non-trivial calculations are attached. A hand-drawn picture of the sample trace (explained below) for the optimal solution is also included with the scratch-work. 
\\ \\ \\
\textbf{Table 1: Optimal Access Times}
\begin{table}[h]
\begin{tabular}{c|c|c|c|c|c|}
  & 1  & 2   & 3   & 4   & 5    \\ \hline
1 & 0.5 & 0.6  & 0.85 & 1.4 & 2.15 \\ \hline
2 &  -  & 0.05 & 0.2  & 0.55 & 1.05 \\ \hline
3 &  -  &  -   & 0.1  & 0.4  & 0.9   \\ \hline
4 &  -  &    - & -    & 0.2  & 0.65  \\ \hline
5 &   - &     -&-     & -    & 0.25 \\ \hline
\end{tabular}
\end{table}
\\ \\
\textbf{Table 2: Optimal Roots}
\begin{table}[h]
\begin{tabular}{c|c|c|c|c|c|}
   & 1     & 2     & 3     & 4     & 5     \\ \hline
1 & $K_1$ & $K_1$ & $K_1$ & $K_1$ & $K_1$ \\ \hline
2 &     -  & $K_2$ & $K_3$ & $K_4$ & $K_4$ \\ \hline
3 &    -   &   -    & $K_3$ & $K_4$ & $K_4$ \\ \hline
4 &     -  &  -     &   -    & $K_4$ & $K_5$ \\ \hline
5 &      - & -      &  -     &  -     & $K_5$   \\ \hline
\end{tabular}
\end{table}
\\ \\
\textbf{Example reconstruction of optimal tree:}
\\
1) You want to find the optimal root for all nodes $K_1$ ... $K_5$, so check the Roots Table at a=1 and b=5. The optimal root is $K_1$ from the table. Add $K_1$. At this point, the left-hand side of node $K_1$ is null (because this is nothing less-than $K_1$ in the given problem) and the right-hand side is the optimal subtree for nodes $K_2$ ... $K_5$.
\\ \\
2) Continue in this manner. You want the optimal root for $K_2$ to $K_5$, so check the Roots Table at a=2 and b=5. The optimal root is $K_4$. Add $K_4$. The left-hand side of the node $K_4$ is the optimal subtree of nodes $K_2$ ... $K_3$ and the right-hand side is from $K_5$ ... $K_5$.
\\ \\
\leavevmode
\end{homeworkProblem}



\pagebreak



\begin{homeworkProblem}
\centerline{\textbf{Problem 5}}
\leavevmode
\\ \\
Here are the generated tables for the given problem. Scratch-work for the non-trivial calculations are attached. A hand-drawn picture of the sample trace (explained below) for the optimal solution is also included with the scratch-work. 
\\ \\ \\
\textbf{Table 1: Optimal Access Times}
\begin{table}[h]
\begin{tabular}{c|c|c|c|c|c|}
  & 1  & 2   & 3   & 4   & 5    \\ \hline
1 & 0.5 & 0.6  & 0.85 & 1.4 & 2.15 \\ \hline
2 &  -  & 0.05 & 0.2  & 0.55 & 1.05 \\ \hline
3 &  -  &  -   & 0.1  & 0.4  & 0.9   \\ \hline
4 &  -  &    - & -    & 0.2  & 0.65  \\ \hline
5 &   - &     -&-     & -    & 0.25 \\ \hline
\end{tabular}
\end{table}
\\ \\
\textbf{Table 2: Optimal Roots}
\begin{table}[h]
\begin{tabular}{c|c|c|c|c|c|}
   & 1     & 2     & 3     & 4     & 5     \\ \hline
1 & $K_1$ & $K_1$ & $K_1$ & $K_1$ & $K_1$ \\ \hline
2 &     -  & $K_2$ & $K_3$ & $K_4$ & $K_4$ \\ \hline
3 &    -   &   -    & $K_3$ & $K_4$ & $K_4$ \\ \hline
4 &     -  &  -     &   -    & $K_4$ & $K_5$ \\ \hline
5 &      - & -      &  -     &  -     & $K_5$   \\ \hline
\end{tabular}
\end{table}
\\ \\
\textbf{Example reconstruction of optimal tree:}
\\
1) You want to find the optimal root for all nodes $K_1$ ... $K_5$, so check the Roots Table at a=1 and b=5. The optimal root is $K_1$ from the table. Add $K_1$. At this point, the left-hand side of node $K_1$ is null (because this is nothing less-than $K_1$ in the given problem) and the right-hand side is the optimal subtree for nodes $K_2$ ... $K_5$.
\\ \\
2) Continue in this manner. You want the optimal root for $K_2$ to $K_5$, so check the Roots Table at a=2 and b=5. The optimal root is $K_4$. Add $K_4$. The left-hand side of the node $K_4$ is the optimal subtree of nodes $K_2$ ... $K_3$ and the right-hand side is from $K_5$ ... $K_5$.
\\ \\
3) Check the table for (2, 3), get the node $K_3$. Add $K_3$ to the left-hand side of $K_4$. The left-hand side of our new $K_3$ node is the optimal subtree of $K_2$ ... $K_2$. Check the table for (5,5) -- obviously it is node $K_5$. Add $K_5$ to the right-hand side of $K_4$. There are no subtrees for node $K_5$.
\\ \\
4) Add $K_2$. Verify that your answer is optimal by adding up the access times * depth and checking Table 1 at a=1 and b=5.
 \\ \\
From these instructions, we have the optimal subtree.
\\ 
\\ \textbf{Please check the attached sheet for a drawn picture representing the above process.}

\end{homeworkProblem}





\end{document}
