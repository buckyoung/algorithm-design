 
%
% Homework Details
%   - Title
%   - Due date
%   - Class
%   - Section/Time
%   - Instructor
%   - Author
%


%
% Basic Document Settings
%

\documentclass{article}

\usepackage{fancyhdr}
\usepackage{extramarks}
\usepackage{amsmath}
\usepackage{amsthm}
\usepackage{amsfonts}
\usepackage{tikz}
\usepackage[plain]{algorithm}
\usepackage[noend]{algpseudocode}
\usepackage{amssymb}

\usetikzlibrary{automata,positioning}


\newcommand*\circled[1]{\tikz[baseline=(char.base)]{
            \node[shape=circle,draw,inner sep=2pt] (char) {#1};}}
            
\topmargin=-0.45in
\evensidemargin=0in
\oddsidemargin=0in
\textwidth=6.5in
\textheight=9.0in
\headsep=0.25in\newcommand{\hmwkClassTime}{Section A}
\linespread{1.1}

\renewcommand\headrulewidth{0.4pt}
\renewcommand\footrulewidth{0.4pt}
\setlength\parindent{0pt}

%
% Create Problem Sections
%

\newcommand{\enterProblemHeader}[1]{
    \nobreak\extramarks{}{Problem \arabic{#1} continued on next page\ldots}\nobreak{}
    \nobreak\extramarks{Problem \arabic{#1} (continued)}{Problem \arabic{#1} continued on next page\ldots}\nobreak{}
}

\newcommand{\exitProblemHeader}[1]{
    \nobreak\extramarks{Problem \arabic{#1} (continued)}{Problem \arabic{#1} continued on next page\ldots}\nobreak{}
    \stepcounter{#1}
    \nobreak\extramarks{Problem \arabic{#1}}{}\nobreak{}
}

\setcounter{secnumdepth}{0}
\newcounter{partCounter}
\newcounter{homeworkProblemCounter}
\setcounter{homeworkProblemCounter}{1}
\nobreak\extramarks{Problem \arabic{homeworkProblemCounter}}{}\nobreak{}

\newenvironment{homeworkProblem}{
    \section{ }
    %\setcounter{partCounter}{1}
    %\enterProblemHeader{homeworkProblemCounter}
}{
    \exitProblemHeader{homeworkProblemCounter}
}



% 
% Header and Footer definition
%

\pagestyle{fancy}
\lfoot{\lastxmark}
\cfoot{$_{Buck}$ $_{Young}$ $_{and}$ $_{Rob}$ $_{Brown}$}


%
% Title Page
%

\title{
    \vspace{2in}
	\textmd{\textbf{\ClassNumber}} \\
    \textmd{\textbf{\ClassName}} \\    
    \normalsize\vspace{0.1in}\small{\hmwkTitle} \\
    \normalsize\vspace{0.1in}\small{Problems \hmwkProblems} \\
	\normalsize\vspace{0.1in}\small{Due \hmwkDueDate}    \\
    \vspace{3in}
}

\author{\textbf{\hmwkAuthorName}}
\date{}

\renewcommand{\part}[1]{\textbf{\large Part \Alph{partCounter}}\stepcounter{partCounter}\\}






% 	% 	%	%	%	%	%
%	Document Start 		%
% 	% 	% 	% 	% 	% 	%

\begin{document}
\pagebreak

\begin{homeworkProblem}
\centerline{\textbf{Problem 18}}
\leavevmode
\\
(6 points) Consider the following problem. The input is a graph G = (V, E), a subset R of vertices of
G, and a positive integer k. The problem is to determine if there is a subset U of V such that
\begin{enumerate}
\item All the vertices in R are contained in U, and
\item the number of vertices in U is at most k, and
\item for every pair of vertices x and y in R, one can walk from x to y in G only traversing vertices that
are in U.
\end{enumerate}
\leavevmode
Show that this problem is NP-hard using a reduction from Vertex Cover
\\ \\
\textbf{Algorithm A Input:} Graph G, subset vertices R, positive integer k
\\ \\ \textbf{Algorithm A Output:} 1 if subset U can be created following the constraints above, 0 otherwise.
\\ \\ \textbf{Theorem:} Algorithm A is NP-hard.
\\ \\ \centerline{\textbf{VertexCover $\leq$ A}}
\\
\\Program VertexCover( H, $l$ ):
\begin{algorithmic}
	\State R = Vertex "S" + Vertex "E$_i$" for all edges in H
	\State G = Every vertex in R + Every vertex in H, connected according to the rules below*
	\State k = number of vertices in R + $l$
	\State return A( G, R, k )
\end{algorithmic}
\leavevmode
\\ \\
Here we want to construct inputs for A which are satisfiable if-and-only-if H and $l$ are satisfiable in VertexCover.
\\ \\
We use the hint to construct an R which includes a vertex S (source vertex) and a bunch of vertices \\ E$_1$, E$_2$, ..., E$_x$ for each edge in H.
\\ \\
Next we construct a G which is essentially R + H without any edges.
\\*Then we add edges to G according to the following rules:
\begin{enumerate}
\item Add an edge from vertex S to every vertex from H 
\item Add an edge from every vertex E$_i$ to that edges terminating vertices from H (Example below)
\end{enumerate}
\leavevmode
Example:
\\ \\ \\ \\ \\ \\ \\ \\
Finally, k is the number of vertices in R plus the number of vertices we were allowed to include in the VertexCover ($l$). This will ensure that the U that is created can choose every vertex in R plus $l$ many vertices in order to make up a vertex cover. 
\\ \\ \\ \\
Here is an example using the above method to construct a valid input for algorithm A. Note that VertexCover will be satisfiable iff algorithm A is satisfiable. 
\\
\end{homeworkProblem}
\pagebreak

\end{document}
