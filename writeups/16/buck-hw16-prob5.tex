 
%
% Homework Details
%   - Title
%   - Due date
%   - Class
%   - Section/Time
%   - Instructor
%   - Author
%


%
% Basic Document Settings
%

\documentclass{article}

\usepackage{fancyhdr}
\usepackage{extramarks}
\usepackage{amsmath}
\usepackage{amsthm}
\usepackage{amsfonts}
\usepackage{tikz}
\usepackage[plain]{algorithm}
\usepackage[noend]{algpseudocode}
\usepackage{amssymb}

\usetikzlibrary{automata,positioning}


\newcommand*\circled[1]{\tikz[baseline=(char.base)]{
            \node[shape=circle,draw,inner sep=2pt] (char) {#1};}}
            
\topmargin=-0.45in
\evensidemargin=0in
\oddsidemargin=0in
\textwidth=6.5in
\textheight=9.0in
\headsep=0.25in\newcommand{\hmwkClassTime}{Section A}
\linespread{1.1}

\renewcommand\headrulewidth{0.4pt}
\renewcommand\footrulewidth{0.4pt}
\setlength\parindent{0pt}

%
% Create Problem Sections
%

\newcommand{\enterProblemHeader}[1]{
    \nobreak\extramarks{}{Problem \arabic{#1} continued on next page\ldots}\nobreak{}
    \nobreak\extramarks{Problem \arabic{#1} (continued)}{Problem \arabic{#1} continued on next page\ldots}\nobreak{}
}

\newcommand{\exitProblemHeader}[1]{
    \nobreak\extramarks{Problem \arabic{#1} (continued)}{Problem \arabic{#1} continued on next page\ldots}\nobreak{}
    \stepcounter{#1}
    \nobreak\extramarks{Problem \arabic{#1}}{}\nobreak{}
}

\setcounter{secnumdepth}{0}
\newcounter{partCounter}
\newcounter{homeworkProblemCounter}
\setcounter{homeworkProblemCounter}{1}
\nobreak\extramarks{Problem \arabic{homeworkProblemCounter}}{}\nobreak{}

\newenvironment{homeworkProblem}{
    \section{ }
    %\setcounter{partCounter}{1}
    %\enterProblemHeader{homeworkProblemCounter}
}{
    \exitProblemHeader{homeworkProblemCounter}
}



% 
% Header and Footer definition
%

\pagestyle{fancy}
\lfoot{\lastxmark}
\cfoot{$_{Buck}$ $_{Young}$ $_{and}$ $_{Rob}$ $_{Brown}$}


%
% Title Page
%

\title{
    \vspace{2in}
	\textmd{\textbf{\ClassNumber}} \\
    \textmd{\textbf{\ClassName}} \\    
    \normalsize\vspace{0.1in}\small{\hmwkTitle} \\
    \normalsize\vspace{0.1in}\small{Problems \hmwkProblems} \\
	\normalsize\vspace{0.1in}\small{Due \hmwkDueDate}    \\
    \vspace{3in}
}

\author{\textbf{\hmwkAuthorName}}
\date{}

\renewcommand{\part}[1]{\textbf{\large Part \Alph{partCounter}}\stepcounter{partCounter}\\}






% 	% 	%	%	%	%	%
%	Document Start 		%
% 	% 	% 	% 	% 	% 	%

\begin{document}
\pagebreak

\begin{homeworkProblem}
\centerline{\textbf{Problem 5}}
\leavevmode
\\
\centerline{\textit{Show that if one of the following three problems has a polynomial time algorithm then they all do.}}
\\ \\
\textbf{CIRCUIT:} The problem is to determine whether a Boolean Circuit (with gates NOT, binary AND, and binary OR) has some input that causes all of the output lines to be 1. Assume that the fan-out (the number of gates that the output of a single gate can be fed into) of the gates in a circuit may be arbitrary.
\\ \\ \textbf{FANIN:} The problem is to determine whether a Boolean Circuit (with gates NOT, arbitrary fan-in AND, and arbitrary fan-in OR), has some input that causes all of the output lines to be 1. Assume that the fan-out (the number of gates that the output of a single gate can be fed into) of the gates in a circuit may be arbitrary.
\\ \\ \textbf{FORMULA:} Determine whether a Boolean formula, with NOT, binary AND, and binary OR operations, is satisfiable.
\\ \\ 
\\ \\
\\
\centerline{\textbf{FANIN $\leq$ CIRCUIT}}
\textbf{Program FANIN(fanin F)} 
\begin{algorithmic}
\State circuit C = F converted to a circuit (according to $\textbf{Transform 1}$)
\\ \Return CIRCUIT(C)
\end{algorithmic}
\leavevmode
\\ \\ \\
\centerline{\textbf{FORMULA $\leq$ FANIN}}
\textbf{Program FORMULA(formula S)} 
\begin{algorithmic}
\State fanin F = S converted to a fanin (according to $\textbf{Transform 2}$)
\\ \Return FANIN(F)
\end{algorithmic}
\leavevmode
\\ \\ \\
\centerline{\textbf{CIRCUIT $\leq$ FORMULA}}
\textbf{Program CIRCUIT(circuit C)} 
\begin{algorithmic}
\State formula S = C converted to a formula (according to $\textbf{Transform 3}$)
\\ \Return FORMULA(S)
\end{algorithmic}
\leavevmode
\\ \\ \\ \\
Thus, since all three problems can be reduced to each other: if one has a polynomial time algorithm, then they all do. 
\\ \\ \\
\centerline{\textit{Transforms are detailed on the following page.}}
\pagebreak
\\ \centerline{\textbf{Transform 1 (fanin to circuit)}}
The goal is to produce a circuit which has maximum fan-in's of 2 for each AND / OR gate. This can easily be accomplished by:
\\ 1) Find any gate where number $N$ of inputs $I$ is greater than 2
\\ 2) Remove this gate and create $N-1$ gates $G$ of the same type
\\ 3) Send the first two inputs $I_1$ and $I_2$ into the first newly created gate ($G_1$)
\\ 4) Send the output of $G_1$ into another newly created gate $G_2$ with the third input $I_3$
\\ 5) Continue in this way for all inputs and then for all gates
\\ $Example:$
\\ \\ \\ \\ \\ \\ \\ \\ \\ 
\\ \centerline{\textbf{Transform 2 (formula to fanin)}}
The goal is to produce a fanin from a formula. This can easily be accomplished by:
\\ 1) Parenthesize every term according to the precedence 1: NOT, 2: AND, 3: OR (if terms have the same precedence, group them together)
\\ 2) Start with the inner-most parens and map to a circuit with arbitrary fan-in's for AND / OR gates
\\ 3) Continue this way until everything is mapped to the fanin
\\ $Example:$
\\ \\ \\ \\ \\ \\ \\ \\ \\
\\ \centerline{\textbf{Transform 3 (circuit to formula)}}
The goal is to produce a formula from a circuit. This can be easily accomplished by:
\\ 1) Starting with earliest gates first, label each output line in a parenthetical way
\\ 2) Continue through for all gates
\\ 3) If the circuit as a whole has multiple fan-outs, simply run them all through AND gates (2 at a time as to not violate the circuit fan-in rule)
\\ $Example:$
\end{homeworkProblem}
\pagebreak

\end{document}
