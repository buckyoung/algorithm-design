 
%
% Homework Details
%   - Title
%   - Due date
%   - Class
%   - Section/Time
%   - Instructor
%   - Author
%


%
% Basic Document Settings
%

\documentclass{article}

\usepackage{fancyhdr}
\usepackage{extramarks}
\usepackage{amsmath}
\usepackage{amsthm}
\usepackage{amsfonts}
\usepackage{tikz}
\usepackage[plain]{algorithm}
\usepackage[noend]{algpseudocode}
\usepackage{amssymb}

\usetikzlibrary{automata,positioning}


\newcommand*\circled[1]{\tikz[baseline=(char.base)]{
            \node[shape=circle,draw,inner sep=2pt] (char) {#1};}}
            
\topmargin=-0.45in
\evensidemargin=0in
\oddsidemargin=0in
\textwidth=6.5in
\textheight=9.0in
\headsep=0.25in\newcommand{\hmwkClassTime}{Section A}
\linespread{1.1}

\renewcommand\headrulewidth{0.4pt}
\renewcommand\footrulewidth{0.4pt}
\setlength\parindent{0pt}

%
% Create Problem Sections
%

\newcommand{\enterProblemHeader}[1]{
    \nobreak\extramarks{}{Problem \arabic{#1} continued on next page\ldots}\nobreak{}
    \nobreak\extramarks{Problem \arabic{#1} (continued)}{Problem \arabic{#1} continued on next page\ldots}\nobreak{}
}

\newcommand{\exitProblemHeader}[1]{
    \nobreak\extramarks{Problem \arabic{#1} (continued)}{Problem \arabic{#1} continued on next page\ldots}\nobreak{}
    \stepcounter{#1}
    \nobreak\extramarks{Problem \arabic{#1}}{}\nobreak{}
}

\setcounter{secnumdepth}{0}
\newcounter{partCounter}
\newcounter{homeworkProblemCounter}
\setcounter{homeworkProblemCounter}{1}
\nobreak\extramarks{Problem \arabic{homeworkProblemCounter}}{}\nobreak{}

\newenvironment{homeworkProblem}{
    \section{ }
    %\setcounter{partCounter}{1}
    %\enterProblemHeader{homeworkProblemCounter}
}{
    \exitProblemHeader{homeworkProblemCounter}
}



% 
% Header and Footer definition
%

\pagestyle{fancy}
\lfoot{\lastxmark}
\cfoot{$_{Buck}$ $_{Young}$ $_{and}$ $_{Rob}$ $_{Brown}$}


%
% Title Page
%

\title{
    \vspace{2in}
	\textmd{\textbf{\ClassNumber}} \\
    \textmd{\textbf{\ClassName}} \\    
    \normalsize\vspace{0.1in}\small{\hmwkTitle} \\
    \normalsize\vspace{0.1in}\small{Problems \hmwkProblems} \\
	\normalsize\vspace{0.1in}\small{Due \hmwkDueDate}    \\
    \vspace{3in}
}

\author{\textbf{\hmwkAuthorName}}
\date{}

\renewcommand{\part}[1]{\textbf{\large Part \Alph{partCounter}}\stepcounter{partCounter}\\}






% 	% 	%	%	%	%	%
%	Document Start 		%
% 	% 	% 	% 	% 	% 	%

\begin{document}
\pagebreak

\begin{homeworkProblem}
\centerline{\textbf{Problem TODO}}
\leavevmode
\\
(4 points) We consider the following medical testing problem. The input consists of:
\\ \\ - A set $A$ of possible diseases
\\ \\ - A collection $T_1$, ... , $T_k$ of medical tests with binary outcomes. Each medical test is a function
from A to \{0, 1\}. If test $T_i$ is applied to someone with a disease a$\in$A, then the medical test
returns the outcome $T_i$(a), which is either 0 or 1. Think of test as being able to distinguish
between a diseases a and b where the tests outcomes are different.
\\ \\ - Integer k
\\ \\
The Dr. Cuddy problem is to determine whether there is a collection S of k tests, such that for all
diseases a and b, there is a test in S that can distinguish between a and b. Show that the Dr. Cuddy
problem is NP-hard by reduction from the triangle problem defined in the previous problem.
\\ \\
\\ \\ \textbf{Theorem:} Problem medical test (MT) is NP-Hard
\\ \\ \centerline{\textbf{Triangle $\leq$ MT}}
\\
\\ To create input for Dr. Cuddy's medical test problem (MT) which returns true if and only if there is a solution to the triangle problem, we will create A (our "tests") to be the union of X, Y, and Z (ie, $A = X\cup Y \cup Z$). We then create our "Tests" ($T_i$) to represent each 3-tuple of our input set W. For each $t\in A$ we set $T_i(t)=1$ if t is in the $i^{th}$ 3-tuple of W, and 0 if it is not.
\\
\\Program Triangle ( subset W, set X, set Y, set Z ):
\begin{algorithmic}[1]
\State A = $X \cup Y \cup Z$ 
\State k = $| W | \le | X \times Y \times Z| = n^3$ 
\For {i=1 to k}
	\For{e in A}
		\If{e in $W_k$} \space\space\space\space\space\space//$W_k$ meaning the $k^{th}$ product-sum (ie, element) in W
			\State $T_i(e) = 1$
		\Else
			\State $T_i(e) = 0$
		\EndIf
	\EndFor
\EndFor
\State \textbf{return} MT( A, $T_1$...$T_k$, k )
\end{algorithmic}
\leavevmode
\\ \\
\end{homeworkProblem}
\pagebreak

\end{document}
