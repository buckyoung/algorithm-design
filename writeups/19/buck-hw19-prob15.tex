 
%
% Homework Details
%   - Title
%   - Due date
%   - Class
%   - Section/Time
%   - Instructor
%   - Author
%


%
% Basic Document Settings
%

\documentclass{article}

\usepackage{fancyhdr}
\usepackage{extramarks}
\usepackage{amsmath}
\usepackage{amsthm}
\usepackage{amsfonts}
\usepackage{tikz}
\usepackage[plain]{algorithm}
\usepackage[noend]{algpseudocode}
\usepackage{amssymb}

\usetikzlibrary{automata,positioning}


\newcommand*\circled[1]{\tikz[baseline=(char.base)]{
            \node[shape=circle,draw,inner sep=2pt] (char) {#1};}}
            
\topmargin=-0.45in
\evensidemargin=0in
\oddsidemargin=0in
\textwidth=6.5in
\textheight=9.0in
\headsep=0.25in\newcommand{\hmwkClassTime}{Section A}
\linespread{1.1}

\renewcommand\headrulewidth{0.4pt}
\renewcommand\footrulewidth{0.4pt}
\setlength\parindent{0pt}

%
% Create Problem Sections
%

\newcommand{\enterProblemHeader}[1]{
    \nobreak\extramarks{}{Problem \arabic{#1} continued on next page\ldots}\nobreak{}
    \nobreak\extramarks{Problem \arabic{#1} (continued)}{Problem \arabic{#1} continued on next page\ldots}\nobreak{}
}

\newcommand{\exitProblemHeader}[1]{
    \nobreak\extramarks{Problem \arabic{#1} (continued)}{Problem \arabic{#1} continued on next page\ldots}\nobreak{}
    \stepcounter{#1}
    \nobreak\extramarks{Problem \arabic{#1}}{}\nobreak{}
}

\setcounter{secnumdepth}{0}
\newcounter{partCounter}
\newcounter{homeworkProblemCounter}
\setcounter{homeworkProblemCounter}{1}
\nobreak\extramarks{Problem \arabic{homeworkProblemCounter}}{}\nobreak{}

\newenvironment{homeworkProblem}{
    \section{ }
    %\setcounter{partCounter}{1}
    %\enterProblemHeader{homeworkProblemCounter}
}{
    \exitProblemHeader{homeworkProblemCounter}
}



% 
% Header and Footer definition
%

\pagestyle{fancy}
\lfoot{\lastxmark}
\cfoot{$_{Buck}$ $_{Young}$ $_{and}$ $_{Rob}$ $_{Brown}$}


%
% Title Page
%

\title{
    \vspace{2in}
	\textmd{\textbf{\ClassNumber}} \\
    \textmd{\textbf{\ClassName}} \\    
    \normalsize\vspace{0.1in}\small{\hmwkTitle} \\
    \normalsize\vspace{0.1in}\small{Problems \hmwkProblems} \\
	\normalsize\vspace{0.1in}\small{Due \hmwkDueDate}    \\
    \vspace{3in}
}

\author{\textbf{\hmwkAuthorName}}
\date{}

\renewcommand{\part}[1]{\textbf{\large Part \Alph{partCounter}}\stepcounter{partCounter}\\}






% 	% 	%	%	%	%	%
%	Document Start 		%
% 	% 	% 	% 	% 	% 	%

\begin{document}
\pagebreak

\begin{homeworkProblem}
\centerline{\textbf{Problem 15}}
\leavevmode
(4 points) Show by reduction that if the decision version of the SAT-CNF problem has a polynomial
time algorithm then the decision version of the 3-coloring problem has a polynomial time algorithm.
\\
Hint: Given a graph, you need to write a Boolean formula in conjunctive normal form that expresses
the fact that the graph is 3 colorable.
\\ \\ \textbf{Theorem:} If SAT-CNF has a polynomial time algorithm, then 3COLOR does as well.
\\ \\ \centerline{\textbf{3COLOR $\leq$ SAT-CNF}}
\\ \\\textbf{3COLOR Input:} Graph G
\\ \\ \textbf{3COLOR Output:} 1 if the vertices of G can be colored with three colors such that no pair of adjacent vertices are colored the same color. 0 otherwise.
\\
\\Program 3COLOR( G ):
\begin{algorithmic}
\State F = G transformed*
\State return SAT-CNF( F )
\end{algorithmic}
\leavevmode
\\ \\
*transformation = We must transform the graph G into a boolean formula F in conjunctive-normal-form (in polynomial time). If the formula F is satisfiable then G can be 3-colored.
\\ \\
We must consider each edge and each connection. Let $_1$, $_2$, and $_3$ represent the three colors. Let E and F be edges in the graph G. We will build up a boolean formula in CNF. 
\\ \\
Note this boolean equivalency (P $\Rightarrow$ Q $\Leftrightarrow$ $\neg$P $\vee$ Q), it is used to convert the 6 conditionals into disjunctions below. 
\\
\begin{algorithmic}
\For{each edge E}
	\State ( $E_1$ $\vee$ $E_2$ $\vee$ $E_3$) $\wedge$ \textit{// The edge must be either colored 1, 2, or 3:}
	\State ( $\neg E_1$ $\vee$ $\neg E_2$ ) $\wedge$ \textit{// If the edge is 1, it cannot be 2}
	\State ( $\neg E_1$ $\vee$ $\neg E_3$ ) $\wedge$ \textit{// If the edge is 1, it cannot be 3}
	\State ( $\neg E_2$ $\vee$ $\neg E_3$ ) $\wedge$ \textit{// If the edge is 2, it cannot be 3}
\EndFor
\\
\For{each pair of adjacent edges (E, F)}
	\State ( $\neg E_1$ $\vee$ $\neg F_1$ ) $\wedge$ \textit{// If E is 1, F cannot be 1}
	\State ( $\neg E_2$ $\vee$ $\neg F_2$ ) $\wedge$ \textit{// If E is 2, F cannot be 2}
	\State ( $\neg E_3$ $\vee$ $\neg F_3$ ) $\wedge$ \textit{// If E is 3, F cannot be 3}
\EndFor
\end{algorithmic}
\leavevmode
\\ \\ \\
This will convert a graph G into a boolean formula F in polynomial time. Therefore, our theorem is proved.
\end{homeworkProblem}
\pagebreak

\end{document}
