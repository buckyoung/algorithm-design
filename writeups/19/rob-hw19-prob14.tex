 
%
% Homework Details
%   - Title
%   - Due date
%   - Class
%   - Section/Time
%   - Instructor
%   - Author
%


%
% Basic Document Settings
%

\documentclass{article}

\usepackage{fancyhdr}
\usepackage{extramarks}
\usepackage{amsmath}
\usepackage{amsthm}
\usepackage{amsfonts}
\usepackage{tikz}
\usepackage[plain]{algorithm}
\usepackage[noend]{algpseudocode}
\usepackage{amssymb}

\usetikzlibrary{automata,positioning}


\topmargin=-0.45in
\evensidemargin=0in
\oddsidemargin=0in
\textwidth=6.5in
\textheight=9.0in
\headsep=0.25in\newcommand{\hmwkClassTime}{Section A}
\linespread{1.1}

\renewcommand\headrulewidth{0.4pt}
\renewcommand\footrulewidth{0.4pt}
\setlength\parindent{0pt}

%
% Create Problem Sections
%

\newcommand{\enterProblemHeader}[1]{
    \nobreak\extramarks{}{Problem \arabic{#1} continued on next page\ldots}\nobreak{}
    \nobreak\extramarks{Problem \arabic{#1} (continued)}{Problem \arabic{#1} continued on next page\ldots}\nobreak{}
}

\newcommand{\exitProblemHeader}[1]{
    \nobreak\extramarks{Problem \arabic{#1} (continued)}{Problem \arabic{#1} continued on next page\ldots}\nobreak{}
    \stepcounter{#1}
    \nobreak\extramarks{Problem \arabic{#1}}{}\nobreak{}
}

\setcounter{secnumdepth}{0}
\newcounter{partCounter}
\newcounter{homeworkProblemCounter}
\setcounter{homeworkProblemCounter}{1}
\nobreak\extramarks{Problem \arabic{homeworkProblemCounter}}{}\nobreak{}

\newenvironment{homeworkProblem}{
    \section{ }
    %\setcounter{partCounter}{1}
    %\enterProblemHeader{homeworkProblemCounter}
}{
    \exitProblemHeader{homeworkProblemCounter}
}



% 
% Header and Footer definition
%

\pagestyle{fancy}
\lfoot{\lastxmark}
\cfoot{$_{Buck}$ $_{Young}$ $_{and}$ $_{Rob}$ $_{Brown}$}


%
% Title Page
%

\title{
    \vspace{2in}
	\textmd{\textbf{\ClassNumber}} \\
    \textmd{\textbf{\ClassName}} \\    
    \normalsize\vspace{0.1in}\small{\hmwkTitle} \\
    \normalsize\vspace{0.1in}\small{Problems \hmwkProblems} \\
	\normalsize\vspace{0.1in}\small{Due \hmwkDueDate}    \\
    \vspace{3in}
}

\author{\textbf{\hmwkAuthorName}}
\date{}

\renewcommand{\part}[1]{\textbf{\large Part \Alph{partCounter}}\stepcounter{partCounter}\\}






% 	% 	%	%	%	%	%
%	Document Start 		%
% 	% 	% 	% 	% 	% 	%

\begin{document}
\pagebreak

\begin{homeworkProblem}
\centerline{\textbf{Problem 14}}
\leavevmode
\\
 (2 points) The input to the three coloring problem is a graph G, and the problem is to decide whether
the vertices of G can be colored with three colors such that no pair of adjacent vertices are colored the
same color. The input to the four coloring problem is a graph G, and the problem is to decide whether
the vertices of G can be colored with four colors such that no pair of adjacent vertices are colored the
same color. Show by reduction that if the four coloring problem has a polynomial time algorithm then
so does the three coloring problem.
\\  \\
\textbf{Input:} A graph G
\\\textbf{Output:} Satisfiability of a 3-coloring of vertices
\\ \\ \\
\textbf{\centerline{3-coloring $\le$ 4-coloring}}
\begin{algorithmic}[1]
\Function{3-coloring}{G}
	\State //Let G' be G with a new vertex, connected to every existing vertex in G	
	\State $G' = G + (v_{new},\; v_1) + ... + (v_{new},\; v_n)$ 
	\State \Return{4-COLORING(G')}
\EndFunction
\end{algorithmic}
\leavevmode
\\ \\
Now, our new vertex $v_{new}$ must take on on of our 4 colors. Since every vertex in G' is adjacent to $v_{new}$, the remaining graph (equivalent to G), must be colored by the remaining 3 colors. See graph illustrating this below.
\end{homeworkProblem}
\pagebreak

\end{document}
