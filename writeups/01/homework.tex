
%
% Homework Details
%   - Title
%   - Due date
%   - Class
%   - Section/Time
%   - Instructor
%   - Author
%

\newcommand{\hmwkTitle}{Greedy Algorithms}
\newcommand{\hmwkProblems}{4 and 5}
\newcommand{\hmwkDueDate}{August 29, 2014}
\newcommand{\ClassName}{Algorithm Design}
\newcommand{\ClassNumber}{CS 1510}
\newcommand{\hmwkAuthorName}{Buck Young and Rob Brown}



%
% Basic Document Settings
%

\documentclass{article}

\usepackage{fancyhdr}
\usepackage{extramarks}
\usepackage{amsmath}
\usepackage{amsthm}
\usepackage{amsfonts}
\usepackage{tikz}
\usepackage[plain]{algorithm}
\usepackage{algpseudocode}
\usepackage{amssymb}

\usetikzlibrary{automata,positioning}


\topmargin=-0.45in
\evensidemargin=0in
\oddsidemargin=0in
\textwidth=6.5in
\textheight=9.0in
\headsep=0.25in\newcommand{\hmwkClassTime}{Section A}
\linespread{1.1}

\renewcommand\headrulewidth{0.4pt}
\renewcommand\footrulewidth{0.4pt}
\setlength\parindent{0pt}

%
% Create Problem Sections
%

\newcommand{\enterProblemHeader}[1]{
    \nobreak\extramarks{}{Problem \arabic{#1} continued on next page\ldots}\nobreak{}
    \nobreak\extramarks{Problem \arabic{#1} (continued)}{Problem \arabic{#1} continued on next page\ldots}\nobreak{}
}

\newcommand{\exitProblemHeader}[1]{
    \nobreak\extramarks{Problem \arabic{#1} (continued)}{Problem \arabic{#1} continued on next page\ldots}\nobreak{}
    \stepcounter{#1}
    \nobreak\extramarks{Problem \arabic{#1}}{}\nobreak{}
}

\setcounter{secnumdepth}{0}
\newcounter{partCounter}
\newcounter{homeworkProblemCounter}
\setcounter{homeworkProblemCounter}{1}
\nobreak\extramarks{Problem \arabic{homeworkProblemCounter}}{}\nobreak{}

\newenvironment{homeworkProblem}{
    \section{ }
    %\setcounter{partCounter}{1}
    %\enterProblemHeader{homeworkProblemCounter}
}{
    \exitProblemHeader{homeworkProblemCounter}
}



% 
% Header and Footer definition
%

\pagestyle{fancy}
\lhead{\ClassNumber\ - \ClassName}
\chead{\hmwkTitle}
\rhead{Problems \hmwkProblems}
\lfoot{\lastxmark}
\cfoot{\thepage}


%
% Title Page
%

\title{
    \vspace{2in}
	\textmd{\textbf{\ClassNumber}} \\
    \textmd{\textbf{\ClassName}} \\    
    \normalsize\vspace{0.1in}\small{\hmwkTitle} \\
    \normalsize\vspace{0.1in}\small{Problems \hmwkProblems} \\
	\normalsize\vspace{0.1in}\small{Due \hmwkDueDate}    \\
    \vspace{3in}
}

\author{\textbf{\hmwkAuthorName}}
\date{}

\renewcommand{\part}[1]{\textbf{\large Part \Alph{partCounter}}\stepcounter{partCounter}\\}






% 	% 	%	%	%	%	%
%	Document Start 		%
% 	% 	% 	% 	% 	% 	%

\begin{document}

\maketitle

\pagebreak

\begin{homeworkProblem}
\centerline{\textbf{Problem 4}}
\leavevmode
\\ \\
\textbf{(a)}
\\
\textbf{Input:} Locations $x_1$, $x_2$, ..., $x_n$ of gas stations along a highway.
\\ \\
\textbf{Output:} Stoppage time at each station which minimizes the total time which you are stopped for gas.
\\ \\
\textbf{Theorem:} The given "next-station" algorithm (NS) is correct / optimal.
\\ \\
\textbf{Proof:} Assume to reach a contradiction that there exists an input on which NS produces an unacceptable output.
\\ \\
Let $S_i$ be the stoppage time at each station
\\ \& NS(I)=\{$S_{\alpha(1)}$, $S_{\alpha(2)}$, ..., $S_{\alpha(n)}$\} be the output of NS on input I
\\ \& OPT(I)=\{$S_{\beta(1)}$, $S_{\beta(2)}$, ..., $S_{\beta(n)}$\} be the optimal solution that agrees with NS(I) the most number of steps
\\ \\ \\ \\ \\ \\
Let $x_k$ be the first stop of the trip upon which NS(I) differs from OPT(I)
\\ \\ By the definition of NS, filling up for any time less than $S_\alpha(k)$ would not get you to the next station and since $S_{\alpha(k)}$ $\neq$ $S_{\beta(k)}$, then we know that $S_{\beta(k)}$ $\textgreater$ $S_{\alpha(k)}$
\\ \\ Let time t be defined as the difference between those stoppage times, t = $S_{\beta(k)}$ - $S_{\alpha(k)}$
\\ \\ Therefore $S_{\beta(k)}$ = $S_{\alpha(k)}$ + t:
\\ \\ \\ \\ \\ \\ \\ \\
Let OPT'(I) = OPT(I) with time t shifted to the next stop: 
\\
\centerline{$S'_k$ = $S_k$ - t}
\centerline{$S'_{k+1}$ = $S_{k+1} + t$}
\\ \\ Clearly, we haven't changed the total stoppage time and we know that we can reach the next stop by the definition of NS. Furthermore we add the difference to the stop at $x_{k+1}$, so we will undoubtably be able to reach the station at $x_{k+2}$ -- and all future stations. 
\\ \\ $\therefore$ We have reached a contradiction because OPT' agrees with NS for one additional step despite OPT being defined as agreeing with NS for the most number of steps.

\end{homeworkProblem}

\pagebreak

\begin{homeworkProblem}
\textbf{(b)}
\\
\textbf{Input:} Locations $x_1$, ..., $x_n$ of gas stations along a highway.
\\ \\
\textbf{Output:} Stoppage time at each station which minimizes the total time which you are stopped for gas.
\\ \\
\textbf{Theorem:} The given "fill-up" algorithm (FU) is not correct or optimal.
\\ \\
\textbf{Proof:}
\\
\\ Consider the input I = $\{x_1, x_2\}$ where $x_2 - x_1$ = $\frac{1}{2} \frac{C}{F}$ 
\\ \\ Note that to reach $x_2$ from $x_1$ it would take $F(x_2 - x_1) = F(\frac{1}{2} \frac{C}{F}) = \frac{1}{2} C$ liters of gas
\\ \\ At $x_1$ the tank is filled to capacity according to FU, taking $\frac{C}{r}$ minutes. But it is possible and correct to reach $x_2$ using only $\frac{1}{2}C$ liters, making the optimal time $\frac{1}{2} \frac{C}{r}$ minutes.
\\ \\ $\therefore$ FU is not correct or optimal. 
\end{homeworkProblem}

\pagebreak

\begin{homeworkProblem}
\centerline{\textbf{Problem 5}}
\leavevmode
\\ \\
\textbf{(a)} 
\\
\textbf{Input:} A set of n points on the real line A where $a_i$ $\in$ $\mathbb{R}$.
\\ \\
\textbf{Output:} Minimum cardinality collection S of unit intervals that cover every point in A.
\\ \\
\textbf{Theorem:} The given "most points" algorithm (MP) is incorrect.
\\ \\
\textbf{Proof:} 
\\ \\
Consider the input P = \{0.5, 1.0, 1.49, 1.51, 2.0, 2.5\}
\\ \\ \\ \\ \\ \\ 
\\ Clearly the interval $I_1$ = (1, 2) covers the most points in P, so MP adds $I_1$ to S.  Now MP has no choice but to add separate intervals $I_2$ = (0.5, 1.5) and $I_3$ = (1.5, 2.5) to cover the remaining two points (0.5 and 2.5, respectively). 
\\ \\ \centerline{\textbf{Thus MP has arrived at S = $\{ I_1,  I_2, I_3\}$}}
\\ \\ \\
Now consider S = \{(0.5, 1.5), (1.5, 2.5)\}
\\ \\ \\ \\ \\ \\ 
Clearly OPT(P) = S, $| OPT(P) | < | MP(P) |$, and MP is not optimal for P.
\\ \\ $\therefore$ MP is not correct.
\\
\pagebreak
\\
\textbf{(b)}
\\
\textbf{Input:} A set of n points on the real line A where $a_i$ $\in$ $\mathbb{R}$.
\\ \\
\textbf{Output:} Minimum cardinality collection S of unit intervals that cover every point in A.
\\ \\
\textbf{Theorem:} The given "left-most" algorithm (LM) is correct / optimal.
\\ \\
\textbf{Proof:} Assume to reach a contradiction that there exists an input on which LM produces an unacceptable output.
\\ \\
Let $I_i$ be unit-intervals 
\\ \& LM(A) = \{$I_{\alpha(1)}$, ..., $I_{\alpha(n)}$\} be the output of LM on input I
\\ \& OPT(A) = \{$I_{\beta(1)}$, ..., $I_{\beta(n)}$\} be the optimal solution that agrees with LM(I) for the most number of steps
\\ \\ \\ \\ \\ \\
Let $a_k$ be the first point where LM(I) differs from OPT(I)
\\ \\ By the definition of LM, we know that the interval which covers $a_k$ is $I_{\alpha(k)}$ = $(a_k, a_k+1)$
\\ \\ In order to define $I_{\beta(k)}$, we should make a few observations:
\\ \\ $\bullet$ All points to the left of $a_k$ are covered by some interval in OPT (by the definition of LM) and $a_k$ is not covered
\\ \\ $\bullet$ The interval $I_{\beta(k)}$ can not start to the right of $a_k$ (because $a_k$ would not be covered) and $I_{\alpha(k)}$ $\neq$ $I_{\beta(k)}$
\\ \\ $\bullet$ Therefore, $I_{\beta(k)}$ must start to the \textbf{left} of $a_k$ in OPT
\\ \\ So for some distance $\Delta$, the interval in OPT which covers point $a_k$ is $I_{\beta(k)}$ = $(a_k-\Delta, a_k+1-\Delta)$
\\ \\ Let OPT' = OPT with $I_{\beta(k)}$ shifted to the right by $\Delta$:
\\ \centerline{$I'_{\beta(k)}$ = $I_{\beta(k)}$ + ($\Delta$, $\Delta$) = $(a_k, a_k+1)$ = $I_{\alpha(k)}$}
\\ \\ As stated earlier, every point $a_i$ where i $\textless$ k has been covered by some previous interval, therefore we can safely move the interval $I_{\beta(k)}$ to the right by a distance $\Delta$. This move may cause a potential overlap to the right, however this overlapping is inconsequential
\\ \\ Further note that OPT' is still optimal because it does not create any additional intervals and covers all the points
\\ \\ $\therefore$ We have reached a contradiction because OPT' agrees with LM for one additional step despite OPT being defined as agreeing with LM for the most number of steps.

\end{homeworkProblem}

\end{document}
