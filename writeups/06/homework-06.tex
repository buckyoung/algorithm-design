 
%
% Homework Details
%   - Title
%   - Due date
%   - Class
%   - Section/Time
%   - Instructor
%   - Author
%

\newcommand{\hmwkTitle}{Greedy Algorithms \& Dynamic Programming}
\newcommand{\hmwkProblems}{16, 2, and 3}
\newcommand{\hmwkDueDate}{Wednesday September 10, 2014}
\newcommand{\ClassName}{Algorithm Design}
\newcommand{\ClassNumber}{CS 1510}
\newcommand{\hmwkAuthorName}{Buck Young and Rob Brown}



%
% Basic Document Settings
%

\documentclass{article}

\usepackage{fancyhdr}
\usepackage{extramarks}
\usepackage{amsmath}
\usepackage{amsthm}
\usepackage{amsfonts}
\usepackage{tikz}
\usepackage[plain]{algorithm}
\usepackage{algpseudocode}
\usepackage{amssymb}

\usetikzlibrary{automata,positioning}


\topmargin=-0.45in
\evensidemargin=0in
\oddsidemargin=0in
\textwidth=6.5in
\textheight=9.0in
\headsep=0.25in\newcommand{\hmwkClassTime}{Section A}
\linespread{1.1}

\renewcommand\headrulewidth{0.4pt}
\renewcommand\footrulewidth{0.4pt}
\setlength\parindent{0pt}

%
% Create Problem Sections
%

\newcommand{\enterProblemHeader}[1]{
    \nobreak\extramarks{}{Problem \arabic{#1} continued on next page\ldots}\nobreak{}
    \nobreak\extramarks{Problem \arabic{#1} (continued)}{Problem \arabic{#1} continued on next page\ldots}\nobreak{}
}

\newcommand{\exitProblemHeader}[1]{
    \nobreak\extramarks{Problem \arabic{#1} (continued)}{Problem \arabic{#1} continued on next page\ldots}\nobreak{}
    \stepcounter{#1}
    \nobreak\extramarks{Problem \arabic{#1}}{}\nobreak{}
}

\setcounter{secnumdepth}{0}
\newcounter{partCounter}
\newcounter{homeworkProblemCounter}
\setcounter{homeworkProblemCounter}{1}
\nobreak\extramarks{Problem \arabic{homeworkProblemCounter}}{}\nobreak{}

\newenvironment{homeworkProblem}{
    \section{ }
    %\setcounter{partCounter}{1}
    %\enterProblemHeader{homeworkProblemCounter}
}{
    \exitProblemHeader{homeworkProblemCounter}
}



% 
% Header and Footer definition
%

\pagestyle{fancy}
\lhead{\ClassNumber\ - \ClassName}
\chead{\hmwkTitle}
\rhead{Problems \hmwkProblems}
\lfoot{\lastxmark}
\cfoot{\thepage}


%
% Title Page
%

\title{
    \vspace{2in}
	\textmd{\textbf{\ClassNumber}} \\
    \textmd{\textbf{\ClassName}} \\    
    \normalsize\vspace{0.1in}\small{\hmwkTitle} \\
    \normalsize\vspace{0.1in}\small{Problems \hmwkProblems} \\
	\normalsize\vspace{0.1in}\small{Due \hmwkDueDate}    \\
    \vspace{3in}
}

\author{\textbf{\hmwkAuthorName}}
\date{}

\renewcommand{\part}[1]{\textbf{\large Part \Alph{partCounter}}\stepcounter{partCounter}\\}






% 	% 	%	%	%	%	%
%	Document Start 		%
% 	% 	% 	% 	% 	% 	%

\begin{document}

\maketitle

\pagebreak




\begin{homeworkProblem}
\centerline{\textbf{Problem 16}}
\leavevmode
\\
\textbf{(a)} 
\textbf{Input:}
\\
\textbf{Output:} 
\\
\textbf{Theorem:} 
\\ \\
\textbf{Proof:} 
\end{homeworkProblem}



\pagebreak



\begin{homeworkProblem}
\textbf{(b)} 
\\
\textbf{Input:}
\\
\textbf{Output:} 
\\ \\
\textbf{Algorithm:}
\\ \\
\textbf{Theorem:} 
\\
\textbf{Proof:} 
\end{homeworkProblem}



\pagebreak



\begin{homeworkProblem}
\textbf{(c)} 
\\
\textbf{Input:}
\\
\textbf{Output:} 
\\ \\
\textbf{Algorithm:}
\\ \\
\textbf{Theorem:} 
\\
\textbf{Proof:} 
\end{homeworkProblem}



\pagebreak



\begin{homeworkProblem}
\centerline{\textbf{Problem 2}}
\leavevmode
\end{homeworkProblem}
The following iterative, array-based, bottom-up algorithm will return the longest sequence $S$ that is a subsequence of three strings $A$, $B$, and $C$ in polynomial ($n^3$) time.
\\ 
\begin{verbatim}
string x, y, z, result = ""
string LCS[A.length][B.length][C.length] = "" // Initialize to empty strings

// Traverse from 1 to size length (leave all a=0, b=0, c=0 as empty strings)
for a = 1 to A.length:
    for b = 1 to B.length:
        for c = 1 to C.length:
        
            if A[a-1] == B[b-1] == C[c-1]: // Access the strings from 0 to length-1
                LCS[a][b][c] = LCS[a-1][b-1][c-1] + A[a-1] // Add the character to the solution
                
                if LCS[a][b][c].length > result.length: // Store iff longest substring
                    result = LCS[a][b][c]
            else:
                x = LCS[a-1][b][c]
                y = LCS[a][b-1][c]
                z = LCS[a][b][c-1]
		
                // Find the maximum substring amongst x, y, and z and add it to the solution	
                if x.length > y.length && x.length > z.length:
                    LCS[a][b][c] = x
                elif y.length > x.length && y.length > z.length:
                    LCS[a][b][c] = y
                else:
                    LCS[a][b][c] = z // All equal or z.length is greatest

return result

\end{verbatim}
\pagebreak



\begin{homeworkProblem}
\centerline{\textbf{Problem 3}}
\leavevmode
The following iterative, array-based, bottom-up algorithm will compute a table for finding the shortest common super-sequence of two strings $A$ and $B$.
\\ 
\begin{verbatim}
SCSS[A.length][B.length] = min(A.length, B.length) // Initialize to shortest string size

for a = 1 to A.length:
    for b = 1 to B.length:
        if A[a-1] == B[b-1]: // Access strings from 0 to length-1
            SCSS[a][b] = SCSS[a-1][b-1] -1
        else:
            SCSS[a][b] = min( SCSS[a-1] )

\end{verbatim}
\textbf{(a)} 
\\ \\
\textbf{(b)}
\\ \\
\textbf{(c)} 
\\ \\
\end{homeworkProblem}



\end{document}